%%%%%%%%%%%%%%%%%%%%%%%%%%%%%%%%%%%%%%%%%%%%%%%%%%%%%%%%%%%
\section{Theory}
\label{sec:theory}
%%%%%%%%%%%%%%%%%%%%%%%%%%%%%%%%%%%%%%%%%%%%%%%%%%%%%%%%%%%

\emph{Using Boundaries: Friction and Boundary Layers}\label{subsec:WallFriction}

Global inputs in the absence of obstacles move a swarm uniformly.  
Shape control requires breaking this uniform symmetry.
The following sections examine using boundaries that stop the particles if they are pushed into the wall.
 These forces are  sufficient to break the symmetry caused by uniform inputs.  
 
 If the $i^{\textrm{th}}$ particle has position $\mathbf{x}_i(t)$ and velocity $\dot{\mathbf{x}}_i(t)$,  we assume the following system model:
 \begin{align}\label{eq:swarmDynamicsAndFric} 
\dot{\mathbf{x}}_i(t)
 &=
 \mathbf{u}(t)
 +F \left( \mathbf{x}_i(t), \mathbf{u}(t) \right), ~i \in [1,n].\\
 F(\mathbf{x}_i(t), \mathbf{u}(t)) &= \begin{cases}
  - \mathbf{u}(t) &\begin{matrix} \mathbf{x}_i(t) \in  \textrm{boundary \textbf{and}}\\
\mathbf{N}(\textrm{boundary$(\mathbf{x}_i(t))$})\cdot   \mathbf{u}(t) \le 0 \end{matrix}
 \\
 0 & \textrm{else} 
 \end{cases}\nonumber
 \end{align}
 Here $\mathbf{N}(\textrm{boundary$(\mathbf{x}_i(t))$})$ is the normal to the boundary at position $\mathbf{x}_i(t)$, and
 $F(\mathbf{x}_i(t), \mathbf{u}(t)) $ is the frictional force provided by the boundary.
 
 
These system dynamics represent particle swarms in low-Reynolds number environments, where viscosity dominates inertial forces and so velocity is proportional to input force~\cite{Purcell1977}. 
 In this regime, the input force command $\mathbf{u}(t)$ controls the velocity of the robots.  
  The same model can be generalized to particles moved by fluid flow where the vector direction of fluid flow $\mathbf{u}(t)$ controls the velocity of particles, or for a swarm of robots that move at a constant speed in a direction specified by a global input $\mathbf{u}(t)$~\cite{Rubenstein2012}.
 Our control problem is to design the control inputs $\mathbf{u}(t)$ to make all $n$ particles achieve a task.