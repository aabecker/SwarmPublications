%%%%%%%%%%%%%%%%%%%%%%%%%%%%%%%%%%%%%%%%%%%%%%%%%%%%%%%%%%%
\section{Torque Control}
\label{sec:theory}
%%%%%%%%%%%%%%%%%%%%%%%%%%%%%%%%%%%%%%%%%%%%%%%%%%%%%%%%%%%
%\subsection{Controlling Torque}


The orientation of an object's major axis is important when a swarm is manipulating a non-symmetric object through narrow corridors. 
Orientation is controllable by applying torque to the object. 
To change the output torque $\tau$ in Eq.~\eqref{eq:torque}, we can choose the direction and magnitude of the force applied $F$, and the moment arm from the object's center of mass (COM) to the point of contact $r$.

\begin{equation}
\tau = F \times r\label{eq:torque}
\end{equation}
The swarm version of \eqref{eq:torque} is the summation of the forces contributed by individual robots.

\begin{align}
\tau_{total} &= \sum\limits_{i=1}^n \rho_i F_i \times (P_i - O )   \label{eq:swarmtorque}\\
F_{total} &= \sum\limits_{i=1}^n \rho_i F_i  \label{eq:swarmforce}
\end{align}

Here $F_i$ is the force that the $i$th robot applies.  If all robots are identical and the control input is uniform, the force is equivalent for every robot and $F_i = F_c$.
Not all robots are in contact with the object.  $\rho_i$ is an indicator variable: $\rho_i$ is 1 if the robot is in direct contact with the object or touching a chain of robots where at least one robot is in contact with the object. Otherwise $\rho_i = 0$.
The moment arm is the robot's position $P_i$ to the object's COM $O=[O_x,O_y]^{\top}$.

\section{Instantiating torque from swarm distribution}
Consider a swarm of robots with probability density $p(x)$. This section examines where to steer the mean of the probability distribution to maximize torque. We examine two problems. First, the torque applied to a rod of length 1 pivoted at 0 when $\theta = 0$ is:
\begin{equation}
\tau_{pivot} = \int_0^1 x\,p(x)\, dx
\end{equation}
Second, the torque applied to a free rod of length 2 from $[-1,0]$ to $[1,0]$, is:
\begin{equation}
\tau_{free} = \int_{-1}^1 x\,p(x)\, dx
\end{equation}
This section considers three canonical probability distributions: uniform, triangular, and normal, all parameterized by mean, $\mu$ and standard deviation, $\sigma$. They are described by:

\begin{figure*}
\centering
\renewcommand{\figwid}{0.66\columnwidth}
\begin{overpic}[width =\figwid]{Uniform.pdf}%\put(1,55){a)}
\end{overpic}
\begin{overpic}[width =\figwid]{Triangular.pdf}
\end{overpic}
\begin{overpic}[width =\figwid]{Normal.pdf}
\end{overpic}
\vspace{-0.5em}
\caption{\label{fig:PDF} Three different distributions are examined in this work: uniform, triangular, and normal.
%\vspace{-2em}
}
\end{figure*}

\begin{figure*}
\centering
\renewcommand{\figwid}{0.66\columnwidth}
\begin{overpic}[width =\figwid]{TorqueUni.pdf}%\put(1,55){a)}
\end{overpic}
\begin{overpic}[width =\figwid]{TorqueTri.pdf}
\end{overpic}
\begin{overpic}[width =\figwid]{TorqueNormal.pdf}
\end{overpic}
\vspace{-0.5em}
\caption{\label{fig:torque} Torque with regard to mean position, $\mu$. Mean position is the pushing location.
%\vspace{-2em}
}
\end{figure*}
%uniform distribution
\begin{align}
p_u(x) &=  \left\{
\begin{array}{ll}
    \frac{1}{2\sqrt{3}\sigma}, &  \textrm{for   } \mu-\sqrt{3}\sigma \leq x \leq \mu+\sqrt{3}\sigma\\
     0, & \textrm{otherwise}\\
\end{array} 
\right.\\
%triangular distribution
p_t(x) &=  \left\{
\begin{array}{ll}
    \frac{x-\mu + \sqrt{6} \sigma}{6\sigma^2}, &  \textrm{for   } \mu-\sqrt{6}\sigma \leq x \leq \mu\\
     \frac{-x+\mu + \sqrt{6} \sigma}{6\sigma^2}, &  \textrm{for   } \mu < x \leq \mu+ \sqrt{6}\sigma\\
     0, & \textrm{otherwise}\\
\end{array} 
\right.\\
%normal distribution
p_n(x) &= \frac{1}{{\sigma \sqrt {2\pi } }}e^{{{ - \left( {x - \mu } \right)^2 } \mathord{\left/ {\vphantom {{ - \left( {x - \mu } \right)^2 } {2\sigma ^2 }}} \right. \kern-\nulldelimiterspace} {2\sigma ^2 }}}
\end{align}

\subsection{Pivoted object torque}
For calculating torque in pivoted object when the swarm has a uniform distribution, we defined $l$ and $u$ for simplicity of the following derivations. Therefore, the torque applied is:
%defining torque uniform distribution
\begin{align}
l &= \max(0,\mu -\sqrt{3} \sigma) \, , u = \min({1,\mu+\sqrt{3}\sigma})\\
\tau_u &= -\frac{l^2}{4\sqrt{3}\sigma}+ \frac{u^2}{4\sqrt{3}\sigma} \textrm{  for    }  u>l
\end{align}
To simplify the following derivations for triangular distribution we use:
%define torque for triangular:
\begin{align}
l_1 &= \max({0,\mu-\sqrt{6}\sigma})\, , l_2 = \max({0,\mu})\\ \nonumber
u_1 &= \min({1,\mu})\, , u_2 = \min({1,\mu+\sqrt{6}\sigma})\\ \nonumber
\tau_t &= \frac{\tau_{left} + \tau_{right}}{36\sigma^2}\nonumber
\end{align}
where $\tau_{left}$ and $\tau_{right}$ are defined as:
\begin{align}
\tau_{left} &=  -2{l_1}^3+3{l_1}^2(\mu-\sqrt{6}\sigma)\\ \nonumber
&+{u_1}^2(2u_1 - 3\mu+3\sqrt{6}\sigma), & \textrm{for     } u_1 > l_1\\ \nonumber
\tau_{right} &= 2{l_2}^3-3{l_2}^2(\mu+\sqrt{6}\sigma)\\ \nonumber
&+{u_2}^2(-2u_2 + 3\mu+3\sqrt{6}\sigma),  & \textrm{for     } u_2 > l_2\\ \nonumber
\end{align}
Also, torque when swarm is normally distributed is defined by:
%defining torque normal distribution
\begin{align} \nonumber
\tau_n &= \frac{(e^{\frac{\mu^2}{2\sigma^2}}-e^{\frac{(1-\mu)^2}{2\sigma^2}})\sigma}{\sqrt{2\pi}}+ \frac{1}{2}\mu(\erf\left(\frac{1-\mu}{\sqrt{2}\sigma}\right)+\erf\left(\frac{\mu}{\sqrt{2}\sigma}\right)) 
\end{align}

Consider robots are uniformly distributed. The mean position that maximizes torque is:
% best place to push for uniform
\begin{align}
\left\{
\begin{array}{ll}
\mu = 1-\sqrt{3}\sigma &   \textrm{for     } \sigma < \frac{1}{2\sqrt{3}}\\
\mu = \sqrt{3}\sigma &   \textrm{otherwise}\\
\end{array} 
\right.
\end{align}

Thus the torque would be:
\begin{align}
\tau_{umax} =\left\{
\begin{array}{ll}
1-\sqrt{3}\sigma &   \textrm{for     } \sigma \leq \frac{1}{2\sqrt{3}}\\
\frac{1}{4 \sqrt{3}\sigma} &   \sigma > \frac{1}{2\sqrt{3}}\\
\end{array} 
\right.
\end{align}
%best place to push for triangular:
For the swarm that has triangularly distributed, the mean position that maximizes torque is:

\begin{align}
\left\{
\begin{array}{ll}
\mu = -\sqrt{6} + \sqrt{12\sigma^2 +1} &   \textrm{for     } 0< \sigma  <\frac{1}{2\sqrt{3}} \\
\mu = \frac{\sqrt{2}}{2} &   \textrm{for     } \sigma \geq \frac{1}{2\sqrt{3}} 
\end{array} 
\right.
\end{align}

Thus the torque would be:
\begin{align}
\tau_{tmax} =\left\{
\begin{array}{ll}
\frac{-1 + \sqrt{1+12\sigma^2}^{\frac{3}{2}}- 18\sigma^3\sqrt{6}}{18\sigma^2} &   \textrm{for     } \sigma \leq \frac{1}{2\sqrt{3}}\\
\frac{\sqrt{2}-2+3\sqrt{6}}{36\sigma^2} &   \sigma > \frac{1}{2\sqrt{3}}\\
\end{array} 
\right.
\end{align}


%best place to push for normal distribution:
For all distributions, the $\mu$ that maximizes torque if $\sigma = 0$ is 1. For the $\tau_{pivot}$ case, the optimal $\mu$ moves in the $-x$ direction as $\sigma$ increases and approaches a limit. We used L'H\^opital's rule and found third derivative of the torque to find the limit when standard deviation goes to infinity. It was because in the first derivative numerator and denominator were both going to zero and we were not able to calculate the limit. The result is:

\begin{align}
\lim_{\sigma\to\infty} \frac{d^3x}{d\mu}(\tau_n)= -1+\frac{3\mu}{2}
\end{align}

which means the best position to push the rod moves left until it reaches $\mu = \frac{2}{3}$.




%%uniform torque
%\begin{align}
%\mu = \textrm{Max}[1-\sqrt{3}\sigma,\sqrt{3}\sigma]
%\end{align}




\subsection{Free object torque}

For calculating torque in free object when the swarm has a uniform distribution, we again defined $l_f$ and $u_f$ for simplicity of the following derivations. Therefore, the torque applied is:
%torque uniform free
\begin{align}
l_f &= \max({-1,\mu -\sqrt{3} \sigma}),\, u_f = \min({1,\mu+\sqrt{3}\sigma})\\
\tau_{uf} &= -\frac{l_f^2}{4\sqrt{3}\sigma}+ \frac{u_f^2}{4\sqrt{3}\sigma} \textrm{  for    }  u_f>l_f
\end{align}

%torque triangular free
To simplify the following derivations for triangular distribution we use:
\begin{align}
l_{f1} &= \max({-1,\mu-\sqrt{6}\sigma}), \,l_{f2} = \max({-1,\mu})\\ \nonumber
u_{f1} &= \min({1,\mu}), \, u_{f2} = \min({1,\mu+\sqrt{6}\sigma}) \nonumber
\end{align}
\begin{align}
\tau_{tf} &=  \left\{
\begin{array}{ll}
\frac{-2{l_{f1}}^3+3{l_{f1}}^2(\mu-\sqrt{6}\sigma)+{u_{f1}}^2(2u_{f1} - 3\mu+3\sqrt{6}\sigma}{36\sigma^2}, &   \textrm{for     } u_{f1} > l_{f1}\\
\frac{2{l_{f2}}^3-3{l_{f2}}^2(\mu+\sqrt{6}\sigma)+{u_{f2}}^2(-2u_{f2} + 3\mu+3\sqrt{6}\sigma}{36\sigma^2}, &   \textrm{for     } u_{f2} > l_{f2}\\
\end{array} 
\right.
\end{align}

%torque normal free
Also, torque when swarm is normally distributed is defined by:
\begin{align} \nonumber
\tau_{nf} &= \frac{\left(-e^{-\frac{(-1+\mu)^2}{2\sigma^2}}+e^{-\frac{-(1+\mu)^2}{2\sigma^2}}\right)\sigma}{\sqrt{2\pi}}\\
 &+ \frac{1}{2}\mu\left(\erf(\frac{1-\mu}{\sqrt{2}\sigma})+\erf(\frac{1+\mu}{\sqrt{2}\sigma})\right) 
\end{align}

%limits and equations



%pivoted comparisons:
\begin{figure*}
\centering
\renewcommand{\figwid}{\columnwidth}
\begin{overpic}[width =\figwid]{BestLocationPivot.pdf}%\put(1,55){a)}
\end{overpic}
\begin{overpic}[width =\figwid]{TorqueFreeBestPush.pdf}
\end{overpic}
\vspace{-0.5em}
\caption{\label{fig:bestLoc} Best location to push in two different situations: when the object is pivoted, and when the object is free.
%\vspace{-2em}
}
\end{figure*}

%\begin{figure}
%\begin{center}
%	\includegraphics[width=0.8\columnwidth]{BestLocationPivot.pdf}
%\end{center}
%\vspace{-1em}
%\caption{\label{fig:besLocPiv}
%Comparison for three different distributions for the best location to push the object.
%}
%\vspace{0em}
%\end{figure}

\begin{figure*}
\centering
\renewcommand{\figwid}{\columnwidth}
\begin{overpic}[width =\figwid]{MaxTorquePivot.pdf}%\put(1,55){a)}
\end{overpic}
\begin{overpic}[width =\figwid]{TorqueFreeComparison.pdf}
\end{overpic}
\vspace{-0.5em}
\caption{\label{fig:maxTorque} Maximum torque possible in two situations: pivoted object and free object.
%\vspace{-2em}
}
\end{figure*}

%\begin{figure}
%\begin{center}
%	\includegraphics[width=0.8\columnwidth]{MaxTorquePivot.pdf}
%\end{center}
%%\vspace{0em}
%\caption{\label{fig:maxTorPiv}
%Comparison for three different distributions for the maximum torque possible.
%}
%\vspace{0em}
%\end{figure}
%
%%free comparisons
%
%\begin{figure}
%\begin{center}
%	\includegraphics[width=0.8\columnwidth]{TorqueFreeComparison.pdf}
%\end{center}
%\vspace{-1em}
%\caption{\label{fig:maxTorqueFree}
%Comparison for three different distributions for the maximum torque possible.
%}
%\vspace{0em}
%\end{figure}


%\begin{figure}
%\begin{center}
%	\includegraphics[width=0.8\columnwidth]{TorqueFreeBestPush.pdf}
%\end{center}
%%\vspace{0em}
%\caption{\label{fig:bestLocFree}
%Comparison for three different distributions for the best location to push the object.
%}
%\vspace{0em}
%\end{figure}

%taking derivative of torque
\begin{align} 
\frac{d\tau}{d\mu} &= -\frac{e^{-\frac{{-1+\mu}^2}{2\sigma^2}}}{\sqrt{2\pi}\sigma} + \frac{1}{2}(\erf(\frac{1-\mu}{\sqrt{2}\sigma})+\erf(\frac{\mu}{\sqrt{2}\sigma})) 
\end{align}

