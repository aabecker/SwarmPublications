Associate Editor's Report:

The Associate Editor (AE) was able to obtain two high-quality reviews
from researchers working directly in this field. After considering
their reviews, it is the opinion of the AE that this paper is not
currently suitable for publication in T-RO. However, the authors are
encouraged to prepare a revised draft, addressing the concerns of the
reviewers.

Please also address how the technique could be practically applied to
navigate inside an actual 3D intestine as opposed to a cross section in
which the particles move on a planar surface. Would magnetic forces be
sufficient to overcome gravity? Would the particles need to be made
neutrally buoyant? 

## add wording about  neutrally buoyant


REVIEWER 1
This paper explains an underactuated control strategy for small teams
of magnetic microparticles which move under the influence of a single
broadcast signal. This underactuated control problem is relevant to
microrobotic swarms moving under magnetic field input. In previous
work, the authors have explored interesting control methods relying on
obstacles in the environment, but the case studied here is more
application-realistic. The control method here is clever and
successful. The work is interesting to read and I think that anyone
with a basic background in controls will find this work interesting.  

The main concept used here to achieve multi-particle control is that
the particles experience a no-slip condition when pushed against the
environment wall. The particles can be removed from the wall by pulling
them. 

The paper is an extension of an IROS2017 paper which introduced the
idea for square workspace only. This TRO submission generalizes to a
non-square workspace, which is a notable improvement. This version of
the work will thus be a helpful addition to the literature. 

The paper gives a very nice review of previous work in using broadcast
signals for multi-particle control. 

Questions to consider:
Is it expected that the no-slip condition will be a strongly satisfied
assumption for in-vivo biomedical applications? In cases where there
were some slip, how could that be handled?

##  To maintain clarity, this paper focused on systems with no-slip boundaries.  While several environments with no slip boundaries exist (e.g. tripe, and intestines with villi), extending to lower friction values is a intriguing avenue that unfortunately does not fit within the 12 page limit for this journal, but is something we are investigating.

This control method requires advance knowledge of the workspace
boundary geometry, in addition to micro-particle state feedback. It
would be interesting to hear about how accurate the boundary geometry
must be known for this control to work well, and perhaps compare this
will medical imaging techniques such as CT or MRI which could
potentially be used to generate such maps. For motion through 3D
lumens, the cross-sectional geometry will change throughout the lumen.

##  add text about systems with large numbers of protrusions or concave workspaces are best handled by motion planners such as RRT.

Abstract: It is stated that �given 3 orthogonal magnetic fields ��.
This statement should be made more precise, because the magnetic
pulling here uses field gradients, not fields. In addition, these
gradients are not orthogonal. An accurate statement would need to be
more complex. 

# made fix


Fig 2: �first contact point� is not clear. There is no way given to
tell where these line up with actual positions on the workspace. 

# FIg 2 is now fig 11.  We updated this figure to clarify (tell what we changed) we rearranged the order because this optimization results....

Fig 3: the meaning of grey areas should be mentioned in the fig or
caption.

# now Figure 2, Gray areas denote regions unaccessible by our motion planner. The particle start positions must be distinct $(\norm{s_2-s_1} \geq \epsilon)$, and the goal positions must be farther than  $\epsilon$  from the boundary, where $\epsilon$ is a small but nonzero user-specified constant.

Figures are hard to read because there is a lot going on, the font is
often too small and text sometimes is obscured.

# We fixed figures ?????

III.C. It�s not clear why this section is relevant. Why is the shortest
path which intersects the wall needed? Some writing structure could
help the reader understand what the goal is here. 

# To better present the motion planner, we moved this section, which covers an optimization result, to the end, and altered the framing text
# TODO: alter the framing text


REVIEWER 6  (1 and 6 )

In this manuscript algorithms to position two particles in arbitrary
locations under uniform actuation are proposed.  The algorithms rely on
differentiating between particles by contacting them with walls or
boundaries of the domain, at which they are assumed to not move unless
actuated in a direction away from the wall.  The paper presents some
results on shortest paths that contact surfaces (but see below on some
presentation issues), then describes the algorithm, which basically
entails moving both particles so that one particle contacts a boundary,
adjusting the relative displacement between particles while the one
particle remains at the boundary, then translating both particles to
the desired location.
Simulations of the algorithm are presented in square and circular
domains, and experiments are described in circular cross sections
inspired by intestinal and stomach scenarios.

Although similar ideas were presented earlier by the authors using
either an obstacle in the workspace or square workspaces, this paper
extends that work to circular and convex polygonal workspaces.	The
overall idea is a novel contribution as other swarm control techniques
focus on either heterogeneity of microrobots or of actuation (such as
fields) rather than distinguishing particles by their proximity to
boundaries.  However, in my opinion the manuscript requires major
revisions before being suitable for publication, mainly to address the
theoretical effectiveness of the algorithm (as detailed below), and
less so to address clarity of presentation.

1.  The main weakness of the paper in my opinion is that the authors
never really prove that their algorithm works, or alternatively how
general their algorithm is, i.e. what are its limits of applicability. 
To be more specific, in their Algorithm 1, if the desired configuration
is not within the 2-move reachable set, then the algorithm targets
instead the closest point in the 2-move reachable set, and then
"iterates until we reach the goal."  When does such iteration actually
achieve the goal?  What are the attainable final goals that can be
reached after iteration?  Does the space of attainable goals depend on
the initial positions of the particles?  While there are simulations,
the ones presented do not explore the entire possible space (which is
understandable since they it is quite large).  I think an analysis of
the attainable space of their algorithm is needed.

# An excellent suggestion!  We rearranged the paper and added section ## which analyzes the reachable set.  We added figures # and # which...

2.  There are issues with the presentation that can be easily improved.
 At the beginning of the paper many terms are not defined which makes
figures and discussion hard to follow. Definitions do come later, but
they should be moved up.  For example, in Fig 2 which symbols are
targets and which are initial conditions are not defined.  In Fig 3
epsilon is shown but not defined nor is its significance explained.  s
and g and \Delta s and \Delta g are not defined. In Fig 12, where is
the goal on the boundary? Can that be indicated on the figures?

## Fig 2 is now 12, and we added
## Figure 3 is now 2, we did ...
## Figure 12 is now
Add captions for start and end (colored squares and circles) to each drawing  (including fig 12)

3.  (more presentation) At the beginning of section IV it wasn't clear
to me whether Algorithm 1 had been described yet when it was first
mentioned, and whether that sentence was a description of what was to
come, or was as assertion that logically followed from the previous
parts of the paper.  

4. A minor comment, sections IIIB and C are a little confusing since it
is not motivated why shortest path is being considered in the overall
argument of the paper.	Furthermore, it seems that shortest path in
IIIB is used for the situation moving two particles, while in IIIC it
is used for the situation of moving only one particle.	Consistency in
the presentation would help.

# we moved these optimization results to ## to better focus on the


