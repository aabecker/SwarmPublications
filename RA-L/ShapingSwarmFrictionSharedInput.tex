%  compress using: gs -sDEVICE=pdfwrite -dCompatibilityLevel=1.4 -dNOPAUSE -dQUIET -dBATCH      -sOutputFile=foo-ShapingSwarmCompressed.pdf ShapingSwarmFrictionSharedInput.pdf

%\documentclass[conference]{IEEEtran}
\documentclass[letterpaper, 10 pt, conference]{ieeeconf}
\IEEEoverridecommandlockouts% This command is only needed if 
                                                          % you want to use the \thanks command
%\overrideIEEEmargins                                      % Needed to meet printer requirements.
\usepackage{times}


\makeatletter 
\let\NAT@parse\undefined
\makeatother

% numbers option provides compact numerical references in the text. 
\usepackage[numbers]{natbib}
\usepackage{multicol}
\usepackage[bookmarks=true]{hyperref}

\usepackage{bbm}
\usepackage{calc}
\usepackage{url}
\usepackage{hyperref}
\hypersetup{
  colorlinks =false,
  urlcolor = black,
  linkcolor = black
}
\usepackage{graphicx}
\usepackage[cmex10]{amsmath}
\usepackage{bm}
\usepackage{amssymb}
\usepackage{rotating}

\usepackage{chngcntr}
\counterwithin{paragraph}{subsection} % makes paragraph depend on subsection


%\usepackage{xfrac}
\usepackage{nicefrac}
\usepackage{cite}
\usepackage[caption=false,font=footnotesize]{subfig}
\usepackage[usenames, dvipsnames]{color}
\usepackage{colortbl}
%\usepackage{caption}

%\usepackage{wrapfig}
\usepackage{overpic}
%\usepackage{subfigure}
%\usepackage{textcomp}
\graphicspath{{./pictures/pdf/},{./pictures/ps/},{./pictures/png/},{./pictures/jpg/}}
\usepackage{breqn} %for breaking equations automatically
\usepackage[ruled]{algorithm}
\usepackage{algpseudocode}
%\usepackage{algorithmic}
\usepackage{multirow}
\usepackage{todonotes}
\usepackage{authblk}

%\newcommand{\todo}[1]{\vspace{5 mm}\par \noindent \framebox{\begin{minipage}[c]{0.98 \columnwidth} \ttfamily\flushleft \textcolor{red}{#1}\end{minipage}}\vspace{5 mm}\par}
% uncomment this to hide all red todos
%\renewcommand{\todo}{}

%% ABBREVIATIONS
\newcommand{\qstart}{q_{\text{start}}}
\newcommand{\qgoal}{q_{\text{goal}}}
\newcommand{\pstart}{p_{\text{start}}}
\newcommand{\pgoal}{p_{\text{goal}}}
\newcommand{\xstart}{x_{\text{start}}}
\newcommand{\xgoal}{x_{\text{goal}}}
\newcommand{\ystart}{y_{\text{start}}}
\newcommand{\ygoal}{y_{\text{goal}}}
\newcommand{\gammastart}{\gamma_{\text{start}}}
\newcommand{\gammagoal}{\gamma_{\text{goal}}}
\providecommand{\proc}[1]{\textsc{#1}}


\newcommand{\ARLfull}{Aero\-space Ro\-bot\-ics La\-bora\-tory }
\newcommand{\ARL}{\textsc{arl}}
\newcommand{\JPL}{\textsc{jpl}}
\newcommand{\PRM}{\textsc{prm}}

\newcommand{\CM}{\textsc{cm}}
\newcommand{\SVM}{\textsc{svm}}
\newcommand{\NN}{\textsc{nn}}
\newcommand{\prm}{\textsc{prm}}
\newcommand{\lemur}{\textsc{lemur}}
\newcommand{\Lemur}{\textsc{Lemur}}
\newcommand{\LP}{\textsc{lp}} 
\newcommand{\SOCP}{\textsc{socp}}
\newcommand{\SDP}{\textsc{sdp}}
\newcommand{\NP}{\textsc{np}}
\newcommand{\SAT}{\textsc{sat}}
\newcommand{\LMI}{\textsc{lmi}}
\newcommand{\hrp}{\textsc{hrp\nobreakdash-2}}
\newcommand{\DOF}{\textsc{dof}}
\newcommand{\UIUC}{\textsc{uiuc}}
%% MACROS


\providecommand{\abs}[1]{\left\lvert#1\right\rvert}
\providecommand{\norm}[1]{\left\lVert#1\right\rVert}
\providecommand{\normn}[2]{\left\lVert#1\right\rVert_#2}
\providecommand{\dualnorm}[1]{\norm{#1}_\ast}
\providecommand{\dualnormn}[2]{\norm{#1}_{#2\ast}}
\providecommand{\set}[1]{\lbrace\,#1\,\rbrace}
\providecommand{\cset}[2]{\lbrace\,{#1}\nobreak\mid\nobreak{#2}\,\rbrace}
\providecommand{\lscal}{<}
\providecommand{\gscal}{>}
\providecommand{\lvect}{\prec}
\providecommand{\gvect}{\succ}
\providecommand{\leqscal}{\leq}
\providecommand{\geqscal}{\geq}
\providecommand{\leqvect}{\preceq}
\providecommand{\geqvect}{\succeq}
\providecommand{\onevect}{\mathbf{1}}
\providecommand{\zerovect}{\mathbf{0}}
\providecommand{\field}[1]{\mathbb{#1}}
\providecommand{\C}{\field{C}}
\providecommand{\R}{\field{R}}
\newcommand{\Cspace}{\mathcal{Q}}
\newcommand{\Uspace}{\mathcal{U}}
\providecommand{\Fspace}{\Cspace_\text{free}}
\providecommand{\Hcal}{$\mathcal{H}$}
\providecommand{\Vcal}{$\mathcal{V}$}
\DeclareMathOperator{\conv}{conv}
\DeclareMathOperator{\cone}{cone}
\DeclareMathOperator{\homog}{homog}
\DeclareMathOperator{\domain}{dom}
\DeclareMathOperator{\range}{range}
\DeclareMathOperator{\sign}{sgn}
\providecommand{\polar}{\triangle}
\providecommand{\ainner}{\underline{a}}
\providecommand{\aouter}{\overline{a}}
\providecommand{\binner}{\underline{b}}
\providecommand{\bouter}{\overline{b}}
\newcommand{\D}{\nobreakdash-\textsc{d}}
%\newcommand{\Fspace}{\mathcal{F}}
\providecommand{\Fspace}{\Cspace_\text{free}}
\providecommand{\free}{\text{\{}\mathsf{free}\text{\}}}
\providecommand{\iff}{\Leftrightarrow}
\providecommand{\subinner}[1]{#1_{\text{inner}}}
\providecommand{\subouter}[1]{#1_{\text{outer}}}
\providecommand{\Ppoly}{\mathcal{X}}
\providecommand{\Pproj}{\mathcal{Y}}
\providecommand{\Pinner}{\subinner{\Pproj}}
\providecommand{\Pouter}{\subouter{\Pproj}}
\DeclareMathOperator{\argmax}{arg\,max}
\providecommand{\Aineq}{B}
\providecommand{\Aeq}{A}
\providecommand{\bineq}{u}
\providecommand{\beq}{t}
\DeclareMathOperator{\area}{area}
\newcommand{\contact}[1]{\Cspace_{#1}}
\newcommand{\feasible}[1]{\Fspace_{#1}}
\newcommand{\dd}{\; \mathrm{d}}
\newcommand{\figwid}{0.22\columnwidth}
\newcommand{\TRUE}{\textbf{true}}
\newcommand{\FALSE}{\textbf{false}}
\DeclareMathOperator{\atan2}{atan2}
\allowdisplaybreaks

\newtheorem{theorem}{Theorem}
\newtheorem{definition}[theorem]{Definition}
\newtheorem{lemma}[theorem]{Lemma}


\pdfinfo{
   /Author (Shiva Shahrokhi, Arun Mahadev, and Aaron T. Becker)
   /Title  (Algorithms For Shaping a Particle Swarm With a Shared Control Input Using Boundary Interaction)
   /CreationDate (D:20160129120000)
   /Subject (Simple Robots)
   /Keywords (Robots;Uniform Control Inputs)
}


% paper title
\title{\LARGE \bf Algorithms For Shaping a Particle Swarm With a\\ Shared Control Input Using Boundary Interaction*}

% You will get a Paper-ID when submitting a pdf file to the conference system
\author{Shiva Shahrokhi, Arun Mahadev, and Aaron T. Becker% <-this % stops a space
\thanks{*This work was supported by the National Science Foundation under Grant No.\ \href{http://nsf.gov/awardsearch/showAward?AWD_ID=1553063}{ [IIS-1553063]}.}% <-this % stops a space
\thanks{Authors are with the Department of Electrical and Computer Engineering,  University of Houston, Houston, TX 77204 USA        {\tt\small  \{sshahrokhi2, aviswanathanmahadev, atbecker\}@uh.edu}}%
}
%\affil{Department of Electrical and Computer Engineering, \\
% University of Houston, Houston, TX 77204-4005 USA\\
% {\tt\small  \{sshahrokhi2, aviswanathanmahadev, atbecker\}@uh.edu}}
%\thanks{S. Shahrokhi, A. Mahadev and  A. Becker are with the Department of Electrical and Computer Engineering,  University of Houston, Houston, TX 77204-4005 USA {\tt\small  \{sshahrokhi2, aviswanathanmahadev, atbecker\}@uh.edu}
%}
%} %\end thanks

\begin{document}



\maketitle
\thispagestyle{empty}
\pagestyle{empty}


\begin{abstract}
Consider a swarm of particles controlled by global inputs. 
This paper presents algorithms for shaping such swarms in 2D using boundary walls.
The range of configurations created by conforming a swarm to a boundary wall is limited. 
We describe the set of stable configurations of a swarm in two canonical workspaces, a circle and a square. 
To increase the diversity of configurations, we add boundary interaction to our model.  
We provide algorithms using friction with walls to place two robots at arbitrary locations in a rectangular workspace.
Next, we extend this algorithm to place $n$ agents at desired locations. 
We conclude with efficient techniques to control the covariance of a swarm not possible without wall-friction. 
Simulations and hardware implementations with 100 robots validate these results.

These methods may have particular relevance for micro- and nano-robots controlled by global inputs.
%, whose small size limits onboard computation and power. For this reason they are usually powered and controlled by global inputs, such as a uniform external electric or magnetic field, and every robot receives the same control inputs.
%Due to their small size, large numbers of micro-robots are required to deliver sufficient payloads.
% Nevertheless, these applications require precision control of the shape and position of the robot swarm. Precision control requires breaking the symmetry caused by the global input.  

 



% KEYWORDS:   uniform control, under-actuation, particle swarm
\end{abstract}

\IEEEpeerreviewmaketitle

%%%%%%%%%%%%%%%
\section{Introduction}\label{sec:Intro}
Particle swarms propelled by an external field, where each particle  receives the same control input, are common in applied mathematics, biology, and computer graphics \cite{Peyer2013,Shirai2005,Chiang2011}.
%
The small size of these robots makes it difficult to perform onboard computation.  Instead, these robots are often controlled by a broadcast signal. 
 The tiny robots themselves are often just rigid bodies, and it may be more accurate to define the robot as the \emph{system} that consists of particles, a uniform control field, and sensing.
  Consider a system of point-particles in a 2D, planar workspace.
Such systems are severely underactuated, having 2 degrees of freedom in the shared planar control input, but $2n$ degrees of freedom for the $n$-particle swarm.
 Techniques are needed that can handle this underactuation. 

 Positioning is a foundational capability for a robotic system, e.g. placement of brachytherapy seeds. 
 In previous work, we showed that the 2D position of each particle in such a swarm is controllable if the workspace contains a single obstacle the size of one particle \cite{AaronManipulation2013}.
 However, requiring a single, small, rigid obstacle suspended in the middle of the workspace is often an unreasonable constraint, especially in 3D.
This paper relaxes that constraint, and provides position control algorithms that only require non-slip wall contacts.
We assume that particles in contact with the boundaries have zero velocity if the uniform control input pushes the particle into the wall.

\begin{figure}
\centering
\vspace{1.5em}
%\begin{overpic}[width=\columnwidth]{firstImage.jpg}\end{overpic}
\begin{overpic}[width=0.45\columnwidth]{firstpicLeft.pdf}\put(28,-10){workspace}\end{overpic}
\begin{overpic}[width=0.45\columnwidth]{magneticsetup.pdf}\put(22,-8){magnetic setup}\end{overpic}
\vspace{1em}
\caption{\label{fig:IntroPic}
Workspace and magnetic setup for an experiment to position particles that receive the same control inputs, but cannot move while a control input pushes them into a boundary.
} \vspace{-1em}
\end{figure}
%\todo{add the picture of magnetic setup}


The paper is arranged as follows. 
After a review of recent related work in Sec.  \ref{sec:RelatedWork},
  Sec.~\ref{sec:theory} introduces a  model for boundary interaction.   
We provide an algorithm to arbitrarily position two particles in Sec.~\ref{sec:PostionControl2Robots},  and two shortest path results for representative workspaces in Sec.~\ref{sec:optimalResults}.
 Section  \ref{sec:simulation} describes implementations of the algorithms in simulation and  Sec.  \ref{sec:expResults} describes hardware experiments, as shown in Fig.~\ref{fig:IntroPic}. 
 We end with directions for future research in Sec.  \ref{sec:conclusion}.

This paper is an elaboration of preliminary work in a conference paper \cite{shahrokhi2017algorithms} which considered only square workspaces. This work extends the analysis to convex workspaces and 3D positioning. This paper also implements the algorithms using a hardware setup inspired by the anatomy of the gastrointestinal tract.



%%%%%%%%%%%%%%%
%%%%%%%%%%%%%%%

%%%%%%%%%%%%%%%%%%%%%%%%%%%%%%%%%%%%%%%%%%%%%%%%%%%%%%%%%%%
\section{Related Work}\label{sec:RelatedWork}
%%%%%%%%%%%%%%%%%%%%%%%%%%%%%%%%%%%%%%%%%%%%%%%%%%%%%%%%%%%


%Unlike \emph{caging} manipulation, where robots form a rigid arrangement around an object~\cite{Sudsang2002,Fink2007}, our swarm of robots is unable to grasp the blocks they push, and so our manipulation strategies are similar to \emph{nonprehensile manipulation} techniques, e.g.~\cite{Lynch1999}, where forces must be applied along the center of mass of the moveable object. 


Robotic manipulation by pushing has a long and successful history\cite{Lynch1999,Lynch1996,Akella2000,Bernheisel2006}.  Key developments introduced the notion of stable pushes and a friction cone.  A \emph{stable push} is a pushing operation by a robot with a flat-plate pushing element in which the object does not change orientation relative to the pushing robot\cite{Lynch1999}.  The \emph{friction cone} is the set of vector directions a robot in contact with an object can push that object with a stable push.
Stable pushes can be used as primitives in an rapidly-expanding random tree to form motion plans.
A key difference is that our robots are compliant and tend to flow around the object, making this similar to fluidic trapping~\cite{Armani2006,Becker2009}.  
%\subsection{Block-pushing and Compliant Manipulation}

While ferrous particles tend to clump in a magnetic field, the magnetotactic bacteria of~\cite{martel2015magnetotactic,ou2012motion} are directed by the orientation of the magnetic field and do not suffer from magnetic aggregation.

Controlling the \emph{shape}, or relative positions, of a swarm of robots is a key ability for a myriad of applications.  Correspondingly, it has been studied from a control-theoretic perspective in  both centralized, e.g. virtual leaders in \cite{egerstedt2001formation}, and decentralized approaches, e.g. decentralized control-Lyapunov function controllers in~\cite{hsieh2008decentralized}. Most approaches assume a level of intelligence and autonomy in the individual robots that exceeds the capabilities of current micro- and nano-robots~\cite{martel2015magnetotactic,Xiaohui2015magnetiteMicroswimmers}.
Instead, this chapter focuses on centralized techniques that apply the same control input to each member of the swarm, as in~\cite{Becker2013b}.



%%%%%%%%%%%%%%%
%%%%%%%%%%%%%%%%%%%%%%%%%%%%%%%%%%%%%%%%%%%%%%%%%%%%%%%%%%%
\section{Global Control Laws for a holonomic swarm}
\label{sec:theory}
%%%%%%%%%%%%%%%%%%%%%%%%%%%%%%%%%%%%%%%%%%%%%%%%%%%%%%%%%%%


% Theory:
%% Models (of each robot, of the global input)
%% Impossibility result
%% Controlling Mean proof
%% Controlling Variance proof (CLF)
%% A Hybrid controller using hysteresis

Emboldened by the three lessons from our online experiments, this section presents automatic controllers for large numbers of particles that only rely on the first two moments of the swarm position distribution.

We represent particles as holonomic robots that move in the 2D plane. We want to control position and velocity of the particles. 
First, assume a noiseless system containing one robot with mass $m$.
 Our inputs are global forces $[u_x,u_y]$. We define our state vector $\mathbf{x}(t)$ as the $x$ position, $x$ velocity, $y$ position and $y$ velocity.
The state-space representation in standard form is: 
\begin{align}\label{eq:stdform}
\dot{\mathbf{x}}(t)  &=  A \mathbf{x}(t) + B \mathbf{u}(t).
\end{align}

and our state space representation is:
\begin{equation}
\begin{bmatrix}
\dot{x}\\ 
\ddot{x}\\
\dot{y}\\
\ddot{y}
\end{bmatrix} = \begin{bmatrix}
0 & 1 & 0 & 0 \\
0 & 0 & 0 & 0\\
0 & 0 & 0 & 1\\
0 & 0 & 0 & 0
\end{bmatrix}  \begin{bmatrix}
x\\
\dot{x}\\
y\\
\dot{y}
\end{bmatrix} + \begin{bmatrix}
0 & 0 \\
\frac{1}{m} & 0 \\
0 & 0 \\
0 & \frac{1}{m}
\end{bmatrix} 
 \begin{bmatrix}
 u_x\\
 u_y\end{bmatrix}.
\end{equation}

We want to find the number of states that we can control, which is given by the rank of the \emph{controllability matrix}
\begin{align}
\mathcal{C} &= [ B, AB, A^2B, ... , A^{n-1}B ].\\
\textrm{Here }
\mathcal{C}&=\left[
\begin{matrix} 
0 & 0\\
\frac{1}{m} & 0 \\
0 & 0 \\
0 & \frac{1}{m}
\end{matrix}
\,\middle\vert\,
\begin{matrix} 
\frac{1}{m}& 0\\
0 & 0\\
0 & \frac{1}{m}\\
0 & 0
\end{matrix}
\,\middle\vert\,
\begin{matrix} 
0 & 0\\
0 & 0 \\
0 & 0 \\
0 & 0
\end{matrix}
\,\middle\vert\,
\begin{matrix} 
0 & 0\\
0 & 0\\
0 & 0\\
0 & 0
\end{matrix}
 \right],\\
 \quad \textrm{rank}(\mathcal{C})&=4,
\end{align}
and thus all four states are controllable. This section starts by proving independent position control of many robots is not possible, but the mean position can be controlled. We then provide conditions under which the variance of many robots is also controllable.

\subsection{Independent control of many particles is impossible}
In this model, a single particle is fully controllable. For holonomic robots, movement in the $x$ and $y$ coordinates are independent, so for notational convenience without loss of generality we will focus only on movement in the $x$ axis. Given $n$ particles to be  controlled in the $x$ axis, there are $2n$ states: $n$ positions and $n$ velocities. Without loss of generality, assume $m=1$.
Our state-space representation is:
\begin{equation}
\begin{bmatrix}
\dot{x}_{1}\\ 
\ddot{x}_{1}\\
\vdots\\
\dot{x}_{n}\\
\ddot{x}_{n}

\end{bmatrix} = \begin{bmatrix}
0 & 1 & \ldots & 0 & 0 \\
0 & 0 & \ldots& 0 & 0 \\
\vdots &  \vdots & \ddots & \vdots & \vdots \\
0 & 0  & \ldots & 0 & 1 \\
0 & 0 & \ldots& 0 & 0 
\end{bmatrix}  \begin{bmatrix}
x_{1}\\
\dot{x}_{1}\\
\vdots \\
x_{n}\\
\dot{x}_{n}
\end{bmatrix} + \begin{bmatrix}
0\\
1\\
\vdots\\
0\\
1
\end{bmatrix} u_x.
\end{equation}
 However, just as with one particle, we can only control two states because the controllability matrix $\mathcal{C}_n$ has rank two:
\begin{equation}
\mathcal{C}_n=\left[ \begin{matrix} 
0\\
1\\
\vdots\\
0\\
1
\end{matrix}
\,\middle\vert\,
  \begin{matrix} 
1\\
0\\
\vdots\\
1\\
0
\end{matrix}
\,\middle\vert\,
\begin{matrix} 
0\\
0\\
\vdots\\
0\\
0
\end{matrix}
\,\middle\vert\,
\begin{matrix} 
0\\
0\\
\vdots\\
0\\
0
\end{matrix}\,\middle\vert\,
\ldots \right], \quad \textrm{rank}(\mathcal{C}_n)=2.
\end{equation}  
\subsection{Controlling the mean position}\label{sec:controlMeanPosition}
This means \emph{any} number of particles controlled by a global command have just two controllable states in each axis. We cannot arbitrarily control the position and velocity of two or more robots, but have options on which states to control.  %One option is to control the position and velocity of the $j^{th}$ robot. To find a potentially more useful option, 
We create the following reduced order system that represents the mean $x$ position and velocity of the $n$ particles:
\begin{align}
\begin{bmatrix}\nonumber
\dot{\bar{x}} \\
\ddot{\bar{x}}
\end{bmatrix} &= \frac{1}{n} \begin{bmatrix}
0& 1& \ldots &0& 1 \\
0& 0& \ldots &0& 0
\end{bmatrix}
\begin{bmatrix}
x_{1}\\
\dot{x}_{1}\\
\vdots\\
x_{n}\\
\dot{x}_{n}
\end{bmatrix} \\
&+ \frac{1}{n}\begin{bmatrix}
0& 0&  \ldots &0& 0 \\
0& 1&  \ldots &0& 1
\end{bmatrix}\begin{bmatrix} 
0\\
1\\
\vdots\\
0\\
1
\end{bmatrix} u_x.
\end{align}
Thus:
\begin{equation}
\begin{bmatrix}
\dot{\bar{x}} \\
\ddot{\bar{x}}
\end{bmatrix} = \begin{bmatrix}
0& 1 \\
0& 0
\end{bmatrix}
\begin{bmatrix}
\bar{x}\\
\dot{\bar{x}}
\end{bmatrix} + \begin{bmatrix} 
0\\
1
\end{bmatrix} u_x.
\end{equation}

We again analyze the controllability matrix $\mathcal{C}_{\mu}$:
\begin{equation}
\mathcal{C}_\mu=\left[ \begin{matrix} 
0\\
1
\end{matrix}
\,\middle\vert\,
 \begin{matrix} 
1\\
0
\end{matrix}
 \right],  \quad \textrm{rank}(\mathcal{C}_\mu)=2.
\end{equation}
Thus the mean position and mean velocity are controllable.


%Due to symmetry of the control input, only the mean position and mean velocity are controllable. However, 
There are several techniques for breaking the symmetry of the control control input to allow controlling more states, for example by using obstacles as in \cite{Becker2013b}, or by allowing independent noise sources as in \cite{beckerIJRR2014}.

We control mean position with a PD controller that uses the mean position and mean velocity. $[u_x,u_y]^\top$ is the global force applied to each robot:
\begin{align}
u_x &= K_{p}(x_{\rm{goal}} - \bar{x}) + K_{d}(0-\dot{\bar{x}}), \nonumber\\
u_y &= K_{p}(y_{\rm{goal}}  - \bar{y}) + K_{d}(0-\dot{\bar{y}}).  \label{eq:PDcontrolPosition}
\end{align}
 $K_{p}$ is the proportional gain, and $K_{d}$ is the derivative gain. 


\subsection{Controlling the variance}\label{sec:VarianceControl}

The variance, $\sigma_x^2,\sigma_y^2$, of $n$ robots' position is computed as:
\begin{align}\label{eq:meanVar}
 \overline{x}(\mathbf{x}) = \frac{1}{n} \sum_{i=1}^n x_{i}, \qquad  %\nonumber \\ 
\sigma_x^2(\mathbf{x}) &= \frac{1}{n} \sum_{i=1}^n (x_{i} - \overline{x})^2,  \nonumber \\ 
 \overline{y}(\mathbf{x}) = \frac{1}{n} \sum_{i=1}^n y_{i}, \qquad  %\nonumber \\ 
\sigma_y^2(\mathbf{x}) &= \frac{1}{n} \sum_{i=1}^n (y_{i} - \overline{y})^2.  
%NOT 1/(N-1) because we are measuring the actual Variance, not the variance of samples drawn from a distribution
\end{align}

Controlling the variance requires being able to increase and decrease the variance.  We will list a sufficient condition for each. 
Microscale systems are affected by unmodelled dynamics. 
These unmodelled dynamics are dominated by Brownian noise, described by~\cite{einstein1956investigations}. 
To model this~\eqref{eq:stdform} must be modified as follows:
\begin{align}
\dot{\mathbf{x}}(t)  &=  A \mathbf{x}(t) + B \mathbf{u}(t) + W \bm{\varepsilon}(t),
\end{align}
where $W\bm{\varepsilon}(t)$ is a random perturbation produced by Brownian noise with magnitude $W$. Given a large obstacle-free workspace with $\mathbf{u}(t)= 0$, a \emph{Brownian noise} process increases the variance linearly with time.
\begin{equation}
\dot{\sigma}_x^2(\mathbf{x}(t), \mathbf{u}(t))  = W \bm{\varepsilon},
\quad \sigma_x^2(t)  = \sigma_x^2(0) + W \bm{\varepsilon} t.
\end{equation}
 If faster dispersion is needed, the swarm can be pushed through obstacles such as a diffraction grating or Pachinko board as in~\cite{Becker2013b}. 

If robots with radius $r$ are in a bounded environment with sides of length $[\ell_x, \ell_y]$, the unforced variance asymptotically grows to the variance of a uniform distribution,
\begin{align}
[\sigma_x^2,\sigma_y^2] = \frac{1}{12}[ (\ell_x - 2 r)^2,(\ell_y - 2 r)^2].\label{eq:VarianceUniformDistribution}
\end{align}

 A flat obstacle can be used to decrease variance. Pushing a group of dispersed robots against a flat obstacle will decrease their variance until the minimum-variance (maximum density) packing  is reached. For large $n$, ~\cite{graham1990penny} showed that the minimum-variance packing  for $n$ circles with radius $r$ is 
 \begin{align} \label{eq:optimalvar}
 \sigma^2_{\rm{optimal}}(n,r) \approx   \frac{\sqrt{3}}{\pi} n r^2\approx 0.55 n r^2.
 \end{align} 
% Sloan minimized second moment U = 1/d^2 \sum_{i=1}^{n} || P_i - P ||^2 = (\sqrt{3} n^2)/(4 \pi), where d = 2r.
% Therefore the variance is d^2 (\sqrt{3} n^2)/(4 \pi), or 4^r^2 (\sqrt{3} n^2)/(4 \pi) = r^2 (\sqrt{3} n^2)/(\pi)If we only need to reduce variance in a single axis, the variance can be reduced to zero, given sufficient space.

We will prove the goal is globally asymptotically stabilizable by using a control-Lyapunov function, as in \cite{Lyapunov1992}.  A suitable Lyapunov function is the squared variance error:
\begin{align}
\label{eq:LyapunovVariance}
V(t,\bf{x})  &= \frac{1}{2} (\sigma^2(\mathbf{x}) - \sigma^2_{\rm{goal}})^2,\nonumber\\
\dot{V}(t,\bf{x}) &= (\sigma^2(\mathbf{x})-\sigma^2_{\rm{goal}})\dot{\sigma}^2(\mathbf{x}).
\end{align}
We note here that $V(t,\mathbf{x})$ is positive definite and radially unbounded, and $V(t,\mathbf{x}) \equiv 0$ only at $\sigma^2(\mathbf{x}) = \sigma^2_{\rm{goal}}$.
To make $\dot{V}(t,\mathbf{x})$ negative semi-definite, we choose
\begin{align}\label{eq:controlVariance}
u(t) &=   \begin{cases}
	 \mbox{move to wall} &\mbox{if } \sigma^2(\mathbf{x})>\sigma^2_{\rm{goal}} \\ 
	 \mbox{move from wall} & \mbox{if } \sigma^2(\mathbf{x}) \le \sigma^2_{\rm{goal}}.
\end{cases} 
\end{align}
 For such a $u(t)$,
 \begin{align}
\dot{\sigma}^2(\mathbf{x}) &=   \begin{cases}
	 \mbox{negative} &\mbox{if } \sigma^2(\mathbf{x})> \max(\sigma^2_{\rm{goal}}, \sigma^2_{\rm{optimal}}(n,r))  \\ 
	 W \epsilon & \mbox{if } \sigma^2(\mathbf{x}) \le \sigma^2_{\rm{goal}},
\end{cases} 
\end{align} and thus
$\dot{V}(t,{\bf x})$ is negative definite and the variance is globally asymptotically stabilizable.% \hfill$\blacksquare$ 



A PID controller to regulate the variance to $\sigma^2_{\rm{ref}}$ is:
\begin{align}
u_x &= K_{p}(x_{\rm{goal}}(\sigma^2_{\rm{ref}}) - \bar{x}) - K_{d}\bar{v}_x + K_{i}(\sigma^2_{\rm{ref}}-\sigma^2_{x}), \nonumber\\
u_y &= K_{p}(y_{\rm{goal}}(\sigma^2_{\rm{ref}})  - \bar{y}) - K_{d}\bar{v}_y + K_{i}(\sigma^2_{\rm{ref}}-\sigma^2_{y}).  \label{eq:PDcontrolVariance}
\end{align}
We call the gain scaling the variance error $K_i$ because the variance, if unregulated, integrates over time.
Eq.~\eqref{eq:PDcontrolVariance} assumes the nearest wall is to the left of the robot at $x=0$, and chooses a reference goal position that in steady-state would have the correct variance according to \eqref{eq:VarianceUniformDistribution}:
\begin{align}
x_{\rm{goal}}(\sigma^2_{\rm{ref}}) = \ell_x/2 = r + \sqrt{3\sigma^2_{\rm{ref}}}.
\end{align}
 If a wall to the right is closer, the signs of $[K_p,K_i]$ are inverted, and the location $x_{\rm{goal}}$ is translated.  


\subsection{Controlling both mean and variance}

The mean and variance of the swarm cannot be controlled simultaneously, however if the dispersion due to Brownian motion is less than the maximum controlled speed, we can adopt the hybrid, hysteresis-based controller shown in Alg.~\ref{alg:MeanVarianceControl} to regulate the mean and variance.  Such a controller normally controls the mean position, but switches to minimizing variance if the variance exceeds some $\sigma_{\rm{max}}^2$. 
 Variance is reduced until less than $\sigma_{\rm{min}}^2$, then control again regulates the mean position. 
 This technique satisfies control objectives that evolve at different rates as in~\cite{kloetzer2007temporal}, and the hysteresis avoids rapid switching between control modes. The process is illustrated in Fig.~\ref{fig:hysteresis}. 


\begin{algorithm}
\caption{Hybrid mean and variance control}\label{alg:MeanVarianceControl}
\begin{algorithmic}[1]
\Require Knowledge of swarm mean $[\bar{x},\bar{y}]$, variance $[\sigma_x^2, \sigma_y^2]$, the locations of the rectangular boundary $\{x_{\rm{min}}, x_{\rm{max}}, y_{\rm{min}}, y_{\rm{max}}\}$, and the target mean position $[x_{\rm{target}},y_{\rm{target}}]$.%TODO: use  \AND, \OR, \XOR, \NOT, \TO, \TRUE, \FALSE \gets
\State $x_{goal} \gets  x_{\rm{target}}$, $y_{\rm{goal}} \gets y_{\rm{target}}$
\Loop
%\State  Compute $\sigma_x^2, \sigma_y^2$

\If {$\sigma_x^2 > \sigma_{\rm{max}}^2$}
\State $x_{\rm{goal}}  \gets x_{\rm{min}}$
\ElsIf { $\sigma_x^2 < \sigma_{\rm{min}}^2$}
\State $x_{\rm{goal}}  \gets  x_{\rm{target}}$
\EndIf

\If{$\sigma_y^2 > \sigma_{\rm{max}}^2$}
\State $y_{\rm{goal}}  \gets y_{\rm{min}}$
\ElsIf { $\sigma_y^2 < \sigma_{\rm{min}}^2$}
\State $y_{\rm{goal}}  \gets  y_{\rm{target}}$
\EndIf
\State Apply \eqref{eq:PDcontrolPosition} to move toward $[x_{\rm{goal}}, y_{\rm{goal}}]$
\EndLoop
\end{algorithmic}
\end{algorithm}


\begin{figure}
\centering
\begin{overpic}[width = 0.8\columnwidth]{hysteresis.pdf}
\put(33,40){$\sigma^2(\mathbf{x}) < \sigma^2_{\rm{min}}$ }
\put(33,15){$\sigma^2(\mathbf{x}) > \sigma^2_{\rm{max}}$}\end{overpic}
%\begin{overpic}[width = 0.45\columnwidth]{VarianceMinMaxBand2v2.pdf}\end{overpic}
\vspace{-0.5em}
\caption{\label{fig:hysteresis}  Hysteresis to control swarm mean and variance. 
%(Right) Switching conditions for variance control are set as a function of $n$, and designed to be larger than the optimal packing density. The plot uses robot radius $r=1/10$.
%\vspace{-2em}
}
\end{figure}

%\begin{figure}
%\centering
%\begin{overpic}[width = 0.5\columnwidth]{VarianceMinMaxBand2v2.pdf}\end{overpic}
%\vspace{-1em}
%\caption{\label{fig:VarianceMinMaxBand} The switching conditions for variance control are set as a function of $n$, and designed to be larger than the optimal packing density. The above plot uses robot radius $r=1/10$.
%}\vspace{-1em}
%\end{figure}


A key challenge is to select proper values for $\sigma_{\rm{min}}^2$ and $\sigma_{\rm{max}}^2$.  The optimal packing variance was given in \eqref{eq:optimalvar}.
The random packings generated by pushing our robots into corners are suboptimal, so we choose the conservative values: 
%shown in Fig.~\ref{fig:hysteresis}:
\begin{align} \label{eq:sigMaxMin}
 \sigma^2_{\rm{min}} &= 2.5r+ \sigma^2_{\rm{optimal}}(n,r), \nonumber\\
  \sigma^2_{\rm{max}} &= 15r+ \sigma^2_{\rm{optimal}}(n,r).
  \end{align}
%\sigma \approx 0.371258 n r,   \sigma_{min} = n*r*1/2, \sigma_{min} = n*r*3/2, 
%  this result was verified in http://www-math.mit.edu/~tchow/penny.pdf

%$\sigma^2_{optimal}(n,r)$ for $n$ circles with radius $r$ is $\approx 
 %    \frac{\sqrt{3}}{\pi} (n r)^2\approx 0.55(n r)^2 $. 
% Sloan minimized second moment U = 1/d^2 \sum_{i=1}^{n} || P_i - P ||^2 = (\sqrt{3} n^2)/(4 \pi), where d = 2r.
% Therefore the variance is d^2 (\sqrt{3} n^2)/(4 \pi), or 4^r^2 (\sqrt{3} n^2)/(4 \pi) = r^2 (\sqrt{3} n^2)/(\pi)If we only need to reduce variance in a single axis, the variance can be reduced to zero, given sufficient space.





%what images should I show here?
%hysteresis control law, \cite{sadra2014}











%%%%%%%%%%%%%%%
%%%%%%%%%%%%%%%



\section{Two Optimal Results}\label{sec:optimalResults}
  Algorithm \ref{alg:optimalAlg} provided a technique to bring two particles to goal positions using global inputs, but did not optimize path length.   Changing the relative positions of particles in any workspace requires making one particle contact the boundary.
  In this section we present two results that can be incorporated into Algorithms \ref{alg:polygonReachbale} and \ref{alg:circularReachbale} to generate shorter motion paths.



 \subsection{Example: Shortest Path in a Square Workspace}\label{subsec:square}
 If the goal configuration cannot be reached in one move but can be reached in three moves, the shortest path has a simple solution. The first move, $m_1$, makes one particle contact a wall, $m_2$ adjusts the relative spacing error  to zero, and $m_3$ takes the particles to their final ositions. 
$m_2$ cannot be shortened, so optimization depends on choosing the location where the particle contacts the wall. 
 Since the shortest distance between two points is a straight line, reflecting the goal position across the boundary wall and plotting a straight line gives the optimal contact location, as shown in Fig. \ref{fig:reflection}. 
  There are four walls, and four candidate solutions, but some candidate solutions may be invalid because a different boundary is hit before the desired first contact position in move $m_1$ (light grey regions) or  invalid because $m_2$ cannot generate the goal relative spacing (dark grey regions).
 %
\begin{figure}
\centering
\begin{overpic}[width=\columnwidth]{Reflection.pdf}\end{overpic}
\vspace{-2em}
\caption{\label{fig:reflection} In a square workspace, the shortest three-move path that reconfigures two particles from starting positions 
 (\protect\tikz \protect\draw[red,fill=white,line width=0.3mm] (-1ex,0) rectangle (0,1ex);,
 \protect\tikz \protect\draw[blue,fill=white,line width=0.3mm] (-1ex,0) rectangle (0,1ex);)
 to goal positions  
 (\protect\tikz \protect\draw[red,fill=white,line width=0.3mm] (0,0) circle (.5ex);,
 \protect\tikz \protect\draw[blue,fill=white,line width=0.3mm] (0,0) circle (.5ex);) are fixed.
 has the property that the incident angle equals the reflected angle, as shown at left. (right) The first contact is colored red if the red particle is the first to touch a boundary, and colored blue if the blue particle is the first to touch.
} \vspace{-1em}
\end{figure}
%

 
 \subsection{Shortest Path in Unit Disk that Intersects Circumference}\label{subsec:circular}

 The shortest path between two points in the unit disk that intersects the circumference is composed of two straight line segments and has an optimal contact point, as shown in Fig.~\ref{fig:shortestpath}. 
 The problem can be simplified by choosing the coordinate system carefully. We define the $x$-axis along the line from the circle center to the starting point: $S=(s,0)$, and define the point of intersection by the angle $\theta$ from the $x$-axis: $P=(\cos \theta,\sin \theta)$. Define the final point $E$ by a radius $e$ and angle $\beta$: $E=e(\cos \beta,\sin \beta)$. Then the length of the two line segments is 
 \begin{align}\scalebox{.8}{$
 \sqrt{{ (s-\cos \theta)^2+\sin^2 \theta}} +  \sqrt{(e \cos \beta-\cos \theta)^2+(e \sin \beta-\sin \theta)^2},$}
 \end{align}
 which is minimized by choosing an appropriate $\theta$ value.
 
%  This equation can be simplified to 
% 
% \begin{equation}
%\sqrt{1+s^2-2 s \cos \theta}+  \sqrt{1+e^2-2 e \cos(\beta-\theta)}. 
% \end{equation}
% 
\begin{figure}
\centering
\renewcommand{\figwid}{\columnwidth}
{\begin{overpic}[width =\figwid]{shortestpath.pdf}
\end{overpic}
}
\caption{\label{fig:shortestpath}{The shortest path between two points $S$ to $E$ in the unit disk that intersects the circumference. The path length as a function of intersection point, $P= (\cos\theta,\sin\theta)$ is shown at right. See \cite{BeckerShortestPath}.}
%\vspace{-1em}
}
\end{figure}

 
 The length of the two line segments as a function of $\theta$ is drawn in the right plot of Fig.~\ref{fig:shortestpath}. There are several simple solutions. If $s$ is 1 or $e$ is 0 or $\beta$ is 0, the optimal angle $\theta^*$ is 0. If $e$ is 1 or $s$ is 0, the optimal angle is $\beta$. Label the origin $O$. 
 The optimal path satisfies the law of reflection off the unit circle, with angle of incidence equal to angle of reflection.
 The angle $\angle{OPS}$ (from the origin to $P$ to $S$) is the same as the angle $\angle{OPE}$ (from the origin to $P$ to $E$). 
 We name these angles $\alpha$. This can be proved by drawing an ellipse whose foci are $S$ and $E$. When the ellipse is tangent to the circle, the point of tangency is $P$. 
  Since the distance from the origin to $P$ is always 1, we can set up three equalities using the law of sines:
 From triangle $OSP$: $\frac{\sin \alpha}{s}=\frac{\sin(\alpha + \theta)}{1}=\frac{\sin \theta}{||SP||}$, and from triangle $OEP$: $\frac{\sin \alpha}{e}=\frac{\sin(\beta - \theta)}{||EP||}$. If we mirror the point $S$ about line $\overline{OP}$ and label this point $C$, from triangle $CEO$: $\frac{\sin(\alpha + \theta)}{e}=\frac{\sin(2 \theta - \beta)}{||CE||}$.
 
 Simplifying this system of equations results in $s=e \csc \theta (s \sin(2 \theta-\beta)+\sin(\beta-\theta))$. Solving this last equation results in a quartic solution that has a closed-form solution with four roots, each of which can be either a clockwise or a counterclockwise rotation $\theta$, depending on the sign of $\beta$, with $-\pi\leq\beta\leq\pi$. We evaluate each and select the solution that results in the shortest length path. %This optimal path satisfies the law of reflection off the unit circle, with angle of incidence equal to angle of reflection. 
 For an interactive Mathematica demonstration of this shortest path, see \cite{BeckerShortestPath}. Because the closed form solution is long, it is included in the paper attachments.
 
 



\section{Simulation}\label{sec:simulation}


%Two simulations were implemented using non-slip contact walls for position control.  The first controls the position of two robots, the second controls the position of $n$ robots.  
\begin{figure*}
\begin{overpic}[width=0.67\columnwidth]{regionMove1.pdf}\put(55.5,26){\begin{turn}{94} 
{\begin{tikzpicture}[thick]
\draw [red,  - , dotted      ] (1,14.0) -- (0,14.0);
\end{tikzpicture}}\end{turn}}\end{overpic}
\begin{overpic}[width=0.67\columnwidth]{regionMove2.pdf}\put(55.5,26){\begin{turn}{94} 
{\begin{tikzpicture}[thick]
\draw [red,   -  , dotted     ] (1,14.0) -- (0,14.0);
\end{tikzpicture}}\end{turn}}\end{overpic}
\begin{overpic}[width=0.67\columnwidth]{regionMove3.pdf}\put(55.5,26){\begin{turn}{94} 
{\begin{tikzpicture}[thick]
\draw [red,   -   , dotted     ] (1,14.0) -- (0,14.0);
\end{tikzpicture}}\end{turn}}\end{overpic}\\

\begin{overpic}[width=0.67\columnwidth]{regionMove4.pdf}\put(55.5,26){\begin{turn}{94} 
{\begin{tikzpicture}[thick]
\draw [red,   -  , dotted      ] (1,14.0) -- (0,14.0);
\end{tikzpicture}}\end{turn}}\end{overpic}
\begin{overpic}[width=0.67\columnwidth]{regionMove5.pdf}\put(55.5,26){\begin{turn}{94} 
{\begin{tikzpicture}[thick]
\draw [red,   -   , dotted     ] (1,14.0) -- (0,14.0);
\end{tikzpicture}}\end{turn}}\end{overpic}
\begin{overpic}[width=0.67\columnwidth]{regionMove6.pdf}\end{overpic}\\

\begin{overpic}[width=0.67\columnwidth]{Move1.pdf}\put(55.5,32.5){\begin{turn}{50.5} 
{\begin{tikzpicture}[thick]
\draw [red,   -   , dotted    ] (1,14.0) -- (0,14.0);
\end{tikzpicture}}\end{turn}}\end{overpic}
\begin{overpic}[width=0.67\columnwidth]{Move2.pdf}\put(55.5,32.5){\begin{turn}{50.5} 
{\begin{tikzpicture}[thick]
\draw [red,   -   , dotted     ] (1,14.0) -- (0,14.0);
\end{tikzpicture}}\end{turn}}\end{overpic}
\begin{overpic}[width=0.67\columnwidth]{Move3.pdf}\put(55.5,32.5){\begin{turn}{50.5} 
{\begin{tikzpicture}[thick]
\draw [red,   -   , dotted    ] (1,14.0) -- (0,14.0);
\end{tikzpicture}}\end{turn}}\end{overpic}\\

\begin{overpic}[width=0.67\columnwidth]{Move4.pdf}\put(55.5,32.5){\begin{turn}{50.5} 
{\begin{tikzpicture}[thick]
\draw [red,   -  , dotted      ] (1,14.0) -- (0,14.0);
\end{tikzpicture}}\end{turn}}\end{overpic}
\begin{overpic}[width=0.67\columnwidth]{Move5.pdf}\put(55.5,32.5){\begin{turn}{50.5} 
{\begin{tikzpicture}[thick]
\draw [red,   -   , dotted     ] (1,14.0) -- (0,14.0);
\end{tikzpicture}}\end{turn}}\end{overpic}
\begin{overpic}[width=0.67\columnwidth]{finalMove.pdf}\end{overpic}
\caption{\label{fig:reachableSet}
Frames from reconfiguring two particles. 
Top six images show a polygonal workspace and the corresponding $\Delta$ configuration space with its 2-move reachable sets. 
Bottom six images show a disk-shaped workspace and the corresponding $\Delta$ configuration space with its 2-move reachable sets. 
For each the move 1 and 3 are simple translations of both particles and so the reachable sets do not change.
The reachable set morphs during move 2 because one particle is held stationary by the boundary. 
See multimedia attachment for animations of each.
}
\end{figure*}

%\subsection{Position Control of Two Robots}
\begin{figure}
\centering
\begin{overpic}[width=0.49\columnwidth]{middlegoalnum.pdf}\put(0,75){a)}\end{overpic}
\begin{overpic}[width=0.49\columnwidth]{middlegoaldist.pdf}\put(0,75){b)}\end{overpic}
\begin{overpic}[width=0.49\columnwidth]{worstnum.pdf}\put(0,75){c)}\end{overpic}
\begin{overpic}[width=0.49\columnwidth]{worstdist.pdf}\put(0,75){d)}\end{overpic}
\caption{\label{fig:contour}
%Plots show performance with one goal on the boundary.
Contour plots showing the number of moves and distance commanded if red particle's goal position is varied in $x$ and $y$. 
Starting positions of red and blue particles  
 (\protect\tikz \protect\draw[red,fill=white,line width=0.3mm] (-1ex,0) rectangle (0,1ex);,
 \protect\tikz \protect\draw[blue,fill=white,line width=0.3mm] (-1ex,0) rectangle (0,1ex);)
 and goal position of blue particle  \protect\tikz \protect\draw[blue,fill=white,line width=0.3mm] (0,0) circle (.5ex); are fixed.
The top row has the blue particle's goal position  at the origin, generating symmetric contour plots.
Moving the  blue particles' goal position  to $(-0.2,0)$, generates non-symmetric contour plots.
}
\end{figure}

\begin{figure}
\centering
%\begin{overpic}[width=\columnwidth]{deltanum.pdf}\end{overpic}\\
%\vspace{1em}
\begin{overpic}[width=\columnwidth]{deltadist.pdf}
\put(65,22){\scriptsize
$\Delta s$
\protect\tikz \protect\draw[myDarkGreen,fill=myDarkGreen,line width=0.3mm] (-1ex,0) rectangle (0,1ex); $=$
\protect\tikz \protect\draw[red,fill=white,line width=0.3mm] (-1ex,0) rectangle (0,1ex); $-$
\protect\tikz \protect\draw[blue,fill=white,line width=0.3mm] (-1ex,0) rectangle (0,1ex);}
\end{overpic}
\vspace{-1em}
\caption{\label{fig:deltanumdist}
The worst-case path length occurs when particles must swap antipodes. This can never be achieved but can be asymptotically approached. Plot shows decreasing error as the number of moves grows.
 Red fit line is $8.66/(\textrm{distance}^3)$, which has an R-squared value of 0.77.
} 
\end{figure}


\begin{figure}
\centering
\renewcommand{\figwid}{1\columnwidth}
{
\begin{overpic}[width =\figwid]{contourDistnew.png}\put(-2,10){\begin{turn}{90} \tiny{unique particles}
\end{turn}}
%
\end{overpic}
\vspace{1em}
\begin{overpic}[width =\figwid]{JustSimulationV6.png}\put(-2,10){\begin{turn}{90} \tiny{unique particles}
\end{turn}}
%
\end{overpic}
\begin{overpic}[width =\figwid]{identical.png}\put(-2,6){\begin{turn}{90} \tiny{interchangeable particles}
\end{turn}}
\end{overpic}
}\caption{\label{fig:contourPlots}{Starting positions of particles $1$ and $2$ 
 (\protect\tikz \protect\draw[blue,fill=white,line width=0.3mm] (-1ex,0) rectangle (0,1ex);,
 \protect\tikz \protect\draw[myMagenta,fill=white,line width=0.3mm] (-1ex,0) rectangle (0,1ex);)
and goal position of particle $2$ (\protect\tikz \protect\draw[myMagenta,fill=white,line width=0.3mm] (0,0) circle (.5ex);) are fixed, and $\epsilon=0.001$.
 The top row of contour plots show the distance if particle $1$'s goal position is varied in $x$ and $y$. The middle row shows the number of moves required for the same configurations. The bottom row shows the same configuration but when the particles are interchangeable.}
\vspace{-1em}
}
\end{figure}
Algorithm \ref{alg:optimalAlg}  was implemented in Mathematica using particles with zero radius. Figure \ref{fig:reachableSet} shows frames of the algorithm in two representative workspaces, square and disk, with two arbitrary starting and goal configurations.
%An online interactive demonstration and source code of the algorithm are available at \cite{Shahrokhi2015mathematicaParticle}.
%  Fig.~\ref{fig:shapeControlMathematica1}  shows  an implementation of this algorithm with robot initial positions represented by hollow squares and final positions by circles. 
 %Dashed lines show the shortest route if robots could be controlled independently, while solid lines show the optimal shortest  path using uniform inputs.
 
 The contour plots in Fig.~\ref{fig:contour} left show the length of the path for two different settings. The top row considers \{$s_1,s_2,g_1$\} = \{$(0.2,0.2),(-0.1,-0.1),(0,0)$\} and the bottom row considers  \{$s_1,s_2,g_1$\} = \{$(0.2,0.2),(-0.1,-0.1),(-0.2,0)$\}, each in a workspace with $r= 0.5$, and $g_2$ ranging over all the workspace. Fig.~\ref{fig:contour} right shows the number of moves and left shows the total distance of the path. This plot shows the nonlinear nature of the path planning. When the goal is in the middle of the workspace, a symmetry in the path length is expected as the top row shows. The bottom row shows a shift in the goal position which breaks the symmetry of the path length in the workspace.
 
%The path length grows when the goals have $\pi$ difference and are very close to the boundary. 
 The worst-case occurs when the ending points are at antipodes along the boundary ($\pi$ angular distance). This can never be achieved but can be asymptotically approached as shown in Fig.~\ref{fig:deltanumdist}. 
 Figure \ref{fig:contourPlots} shows the same concepts in a square workspace. Figure \ref{fig:contourPlots} top and middle row considers the particles for three arbitrary starting and goal positions for the particles. 

 Thus far, this paper has considered the particles to be unique. If particles are interchangeable, the path lengths often decrease, which can be computed by running Alg.~\ref{alg:optimalAlg} twice,  but swap the goal positions for the second run and select the shortest path.  The bottom row of  Fig.~\ref{fig:contourPlots} considers interchangeable particles with the same configuration as the middle row with unique particles. The worst-case path lengths decrease by 33\%, 60\%, and 30\% for the three test cases shown. %TODO: shiva, insert numbers.
 
 %If the length of each side of the square workspace is $L$, the worst-case path length is $(\sqrt{2}+2)L$.
 
% The plots in Fig.~\ref{fig:deltanumdist} show the exponentially increasing number of moves and distance when the accuracy of reaching to the goal ($\delta$) is getting to zero when the goal positions have $\pi$ difference with each on the boundaries.











%%%%%%%%%%%%%%%
%%%%%%%%%%%%%%%

%%%%%%%%%%%%%%%%%%%%%%%%%%%%%%%%%%%%%%%%%%%%%%%%%%%%%%%%%%%
\section{Object manipulation results}\label{sec:exp}
%%%%%%%%%%%%%%%%%%%%%%%%%%%%%%%%%%%%%%%%%%%%%%%%%%%%%%%%%%%

This section analyzes an \emph{object manipulation} task attempted by both our hybrid, hysteresis-based controller, inspired by the analysis of human users in Section~\ref{sec:expResults}.  
The swarm must deliver the object to the goal region.  To solve this object manipulation task, we divide the task into two components: designing a policy for the object, and designing a control law for the swarm to push the object according to the policy.

\subsection{Learning a policy for the object}\label{subsec:objectpolicy}

To design the policy, we first discretize the environment. 
In \cite{ShahrokhiIROS2015}, we used breadth-first search (BFS) on this discretized grid to determine $\mathbf{M}_{BFS}$, the shortest distance from any grid cell to the goal, and generated a gradient map $\nabla \mathbf{M}$  shown in Fig.~\ref{fig:BFSGradient}.  
The object's center of mass is at $\mathbf{b}$ and has average radius $r_b$, so we define the desired direction for the object as $\mathbf{D}(\mathbf{b}) = \nabla \mathbf{M}(\mathbf{b})$. 
The robots were directed behind the object at  $\mathbf{b} - 1.5 r_b \mathbf{D}(\mathbf{b})$, then directed to  $\mathbf{b} - 0.1 r_b \mathbf{D}(\mathbf{b})$ to push the object toward the goal location. However, BFS has no penalties on object collisions with obstacles, and the resulting collisions made the BFS solution slow.

 To avoid obstacles, this paper models object movement as a Markov Decision Process (MDP) with non-deterministic movement.  
  Value iteration,  as described in \cite{Thrun2005}, is used to learn an \emph{optimal policy}: a function that assigns a control input to every possible state.
 At each state the object can be commanded to move in one of eight directions with a small probability of moving in the wrong direction. 
 
 A corresponding reward function gives a high reward to the goal state,  
 a large negative reward to states including obstacles, and a small negative reward to all other states.
The reward function $r(x,u)$ is defined as
\begin{align}
r(x,u) &=  \left\{
\begin{array}{ll}
     +100, &  \textrm{if } u \textrm{ leads to goal state}\\
      -100, & \textrm{if } u \textrm{ leads to a state with an obstacle} \\
      -1, & \textrm{otherwise}\\
\end{array} 
\right.
\end{align}
where $u$ is the action and $x$ is the current state. 
  Value iteration iteratively learns the value $\hat{V}(x)$ for the object being in each state $x$. The optimal policy is the control action that, in probability, results in the largest value:
   \begin{align} \mathbf{D}(x) = \arg\max_u   [ r(x,u) + \sum\limits_{j=1}^N \hat{V}(x_j) p(x_j| u,x)].  \label{eq:OptimalPolicy}
   \end{align}
   The value function $\hat{V}(x_j) $ is calculated by computing the value $\hat{V}$ for all $N$ states and iterating until convergence:
\begin{align}
\text{for }&\text{$i=1$ to $N$ do} \nonumber \\
&\hat{V} (x_j) = \gamma \max_u [r(x_j,u) + \sum\limits_{i=1}^N \hat{V}(x_i) p(x_i| u,x_j)] \label{eq:ValueIteration}\\
\text{end}& \nonumber
\end{align}
In our experiments $\gamma = 0.97$, and \eqref{eq:ValueIteration} was iterated 200 times. A {\sc Matlab} implementation of this algorithm is available at \href{https://www.mathworks.com/matlabcentral/fileexchange/49992-mdp-robot-grid-world-example}{\emph{MDP grid world example}}.%\todo{link to algorithm}.  
%Rather than a shortest distance to goal, $\nabla \mathbf{M}_{MDP}$  gives the approximate minimum cost to goal.

$\mathbf{M}_{BFS}$ and the value function are shown in Fig.~\ref{fig:BFSGradient}. 
In 10 simulations with 100 robots, pushing the object to goal using BFS required a mean and standard deviation of $183\pm 179$ s while Policy Iteration required $90\pm 35$ s.
%\todo{(report results of 10 simulations with BFS and with PI.  Give mean and std)}

\begin{figure}
\centering
\begin{overpic}[scale=0.19]{GradientView.png}
\end{overpic}
\begin{overpic}[scale=0.19]{PolicyIter.png}
\end{overpic}
\begin{overpic}[scale=0.262]{MDPmap.pdf}
\end{overpic}
\vspace{-0.5em}
\caption{\label{fig:BFSGradient}The BFS algorithm finds the shortest path for the moveable object  to compute gradient vectors (left). Modeling as an MDP enables encoding penalties for being near obstacles. (Middle) The control policy from value iteration. (Right) The vision algorithm detects obstacles, used to produce the
value function and control policy for the hardware setup. 
%\vspace{-2em}
}
\end{figure}


\subsection{Potential fields for swarm management with a compliant manipulator}

Unfortunately, when the swarm is in front of the object, control law \eqref{eq:PDcontrolPosition} pushes the object backwards.  To fix this, we implement a potential field approach \cite{spong2008robot} that attracts the swarm to the intermediate goal, but repulses the swarm from in front of the object.
The repulsive potential field is centered at the object's COM and is active for a radius $\rho_0$, but is implemented only when the swarm mean is within $\theta$ of the desired direction of motion $\mathbf{D}$ as shown in Fig.~\ref{fig:potentialField}.
\begin{align}
F_{att} &= -\zeta \Delta \rho / \rho \\
%\mathbf{u} &= [\bar{x},\bar{y}] - \mathbf{b} \nonumber \\
\phi &=\cos^{-1}\left( \frac{ \mathbf{D} \cdot ( [\bar{x},\bar{y}] - \mathbf{b}) }{\norm{\mathbf{D}} \norm{ [\bar{x},\bar{y}] - \mathbf{b}} } \right) \\
%\theta &= \mathrm{angularDistance}\left( \alpha, \beta \right)\nonumber \\
 F_{rep} &=  \left\{
\begin{array}{ll}
      \eta( 1/\rho- 1/\rho_0) \frac{1}{\rho^2} \Delta \rho, ~& \rho\leq \rho_0~\&~\phi <  \theta \\
      0, & \textrm{ otherwise} \\
\end{array} 
\right.\\
F_{pot} &= F_{att} + F_{rep} \label{eq:potentialfield}
\end{align}

In simulations, $\theta =  \pi/2$,  $\eta  = 75$, $\zeta = 2$ and $\rho_0 = 3$. Because the kilobot hardware experiments have a slower time constant, they use $\theta =  \pi/2$,  $\eta  = 50$, $\zeta = 1$ and $\rho_0 = 7.5$. 

In 10 simulations with 100 robots, pushing the object to goal without a repulsive potential field failed in two of twelve runs. No failures occurred with the repulsive potential field.  Of successful trials, completion time without repulsive potential fields required $\mu=245, \sigma=135$ s while using repulsive potential fields required $\mu=90, \sigma=35$ s.
%In 10 simulations with 100 robots, completion time without repulsive potential fields required $\mu=245, \sigma=135$s while using repulsive potential fields required $\mu=90, \sigma=35$.

%\todo{report results of 10 simulations with and without Potential fields.  Give mean and std}

\begin{figure}
\centering
\begin{overpic}[width=1\columnwidth]{PotentialField.pdf}\end{overpic}
%\todo{I like the 'target' symbol, but it is not self-documenting.  We need a legend explaining the min and max variance ellipses, the goal region, the variance, the mean, the object COM, and the target mean position.  I think these are easiest to make in powerpoint.
%Please use the same color and line style for the variance min and max as you use in Figure 4.
%}
%{blockpushingImageWithMeanAndVarianceOverlay.png}
\caption{\label{fig:potentialField} (Left) The attractive field is centered behind the object's COM. (Middle) The repulsive field is centered at the object's COM. (Right) Combining these forces prevents the swarm from pushing the object backwards.}
\end{figure}

\subsection{Outlier rejection}\label{subsec:OutlierRejection}

The variance controller in Alg.~\ref{alg:MeanVarianceControl} is a greedy algorithm that is susceptible to outliers. The controller in \cite{Shahrokhi2015} failed in $14\%$ trials, often because workspace obstacles made some robots unable to reach the object. This failure rate increases if  object weight increases or ground-robot friction increases. The mean and covariance calculations \eqref{eq:meanVar} included all robots in the workspace. Robots that cannot reach the object due to obstacles skew these calculations. The state machine in Fig.\ \ref{fig:Region}.a solves this problem by creating two states for the maze: either main or transfer. Each state has a set of regions representing a discretized visibility polygon. Whenever the object crosses a region boundary the state toggles. The main regions are generated by extending obstacles until they meet another obstacle shown in Fig.~\ref{fig:Region}.b. The transfer regions are perpendicular to obstacle boundaries, and act as a buffer between two main regions shown in Fig.~\ref{fig:Region}.c.
This filtering increases experimental success because the mean calculation only includes nearby robots that can directly interact with the object. In the example, we want the robots to push the object to the right. Without filtering the robots, the orange star is the mean and the algorithm would instruct the robots to push the object southeast. The filtered mean is at the yellow star and the algorithm instructs the robots to push the object directly east. 
%When the object leaves main region 1 the maze state switches to transfer. 
%The object is in transfer region 0, so only robots in transfer region 0 are included in the mean and covariance calculations.  
%This heuristic improves performance by $50\%$ or less regarding the object heaviness.
In 10 simulations with 100 robots, completion time without outlier rejection required $\mu=245, \sigma=135$ s while using outlier rejection required $\mu=90, \sigma=35$ s.
%\todo{report results of 10 simulations with and without regions.  Give mean and std}


\begin{figure*}
\begin{center}
	\begin{overpic}[width=0.45\columnwidth]{mainRegions.pdf}\end{overpic}
	\begin{overpic}[width=0.45\columnwidth]{transferRegions.pdf}\end{overpic}\\
	\vspace{0.5em}
	\begin{overpic}[width=0.6\columnwidth]{stateMachine.pdf}\end{overpic}
\end{center}
\caption{\label{fig:Region}  Outlier rejection state machine and regions.
}
\end{figure*}

\subsection{Algorithm for object manipulation}
 We use the hybrid hysteresis-based controller in Alg.~\ref{alg:MeanVarianceControl}  to track the desired position, while maintaining sufficient robot density to move the object by switching to minimize variance whenever variance exceeds a set limit. The minimize variance control law \eqref{eq:PDcontrolVariance} is slightly modified to choose the nearest corner further from the goal than $\mathbf{b}$ with an obstacle-free straight-line path to $\mathbf{b}$. 
The control algorithm  for object manipulation is listed in Alg.~\ref{alg:BlockPushing}. 

In rare cases during simulations the swarm may become trapped in a local minima of \eqref{eq:potentialfield}.
If the swarm mean position does not change for ten seconds, the swarm is assumed to be in a local minima and is commanded to move toward the previous corner. As soon as the mean position changes, normal control resumes.

\begin{wrapfigure}{R}{0.3\textwidth}
  \vspace{-20pt}
  \begin{center}
\begin{overpic}[width=0.3\columnwidth]{AllShapes.pdf}\end{overpic}
  \end{center}
  \vspace{-1em}
\caption{\label{fig:Shapes} Six equal-area objects were tested. %Six object shapes were tested.
\vspace{-1em}
}
\end{wrapfigure}



%\todo{add the timeout function: I added two lines.}

\begin{algorithm}
\caption{Object-manipulation controller for a robotic swarm.}\label{alg:BlockPushing}
\begin{algorithmic}[1]
\Require Knowledge of moveable object's center of mass $\mathbf{b}$; swarm mean $[\bar{x},\bar{y}]$ and variance $[\sigma_x^2, \sigma_y^2]$, each calculated using the regions function from \S \ref{subsec:OutlierRejection};  map of the environment
\State Compute optimal policy for object, according to \S \ref{subsec:objectpolicy}
\While{$\mathbf{b}$ is not in goal region}
\State $\sigma^2 \gets \max{(\sigma_x,\sigma_y)}$
\If {$\sigma^2 > \sigma_{max}^2$}
\While{$\sigma^2 > \sigma_{min}^2$}
\State $ [x_{goal}, y_{goal}] \gets $ the nearest corner in current region
\State Apply \eqref{eq:PDcontrolPosition} to move toward $[x_{goal}, y_{goal}]$
\EndWhile
\Else  
%\If {$\mathrm{distance}( \mathbf{b}, [x_{goal}, y_{goal}] ) > k_1 r_b$}
%	\State$r_p \gets k_2 r_b$  \Comment{guarded move}
%	\Else
%	\State$r_p \gets k_3 r_b$  \Comment{pushing move}
%	\EndIf
%\State$r_p \gets k_3 r_b$  \Comment{pushing move}
\State Calculate $\mathbf{D}(\mathbf{b})$  \Comment{direction for object at $\mathbf{b}$}
\State Apply \eqref{eq:potentialfield}   \Comment{potential field to push object in correct direction}
\EndIf
\EndWhile
\end{algorithmic}
\end{algorithm}


\subsection{Automated object manipulation (simulation)}
Fig.~\ref{fig:story} shows snapshots during an execution of this algorithm in simulation. 
Experimental results of parameters sweeps are summarized in Fig.~\ref{fig:AutoVeryParam}.  Each trial measured the time to deliver the object to the goal location.  The default parameter settings used 100 robots, a normalized weight of 1, a hexagon shape, and Brownian noise (applied each simulation step) with $W=5$.  

\begin{figure*}
\centering
%\renewcommand{\figwid}{0.19\columnwidth}
%\href{http://youtu.be/tCej-9e6-4o}{\begin{overpic}[width =\figwid]{story1.png}\put(6,15){T = 5 s}
%\end{overpic}
%\begin{overpic}[width =\figwid]{story2.png}\put(6,15){T = 12 s}
%\end{overpic}
%\begin{overpic}[width =\figwid]{story3.png}\put(6,15){T = 20 s}
%\end{overpic}
%\begin{overpic}[width =\figwid]{story4.png}\put(6,15){T = 25 s}
%\end{overpic}
%\begin{overpic}[width =\figwid]{story5.png}\put(6,15){T = 33 s}
%\end{overpic}}
\begin{overpic}[width =\columnwidth]{SwarmRun.pdf}
\end{overpic}
\vspace{-2em}
\caption{\label{fig:story}\href{http://youtu.be/tCej-9e6-4o}{Snapshots showing an object manipulation simulation with 100 robots under automatic control.  See animation in~\cite{ShivaVideo2015}.}
%\vspace{-2em}
}
\end{figure*}

The interaction between the robots and object is impulsive so, like the study of impulsive pulling in  \cite{christensen2016let},  adding additional robots decreases completion time, but with diminishing returns. 
 After 75 robots, additional robots no longer can interact with the object and do not contribute to the task success. 
Brownian noise adds stochasticity.  This randomness can break the object free if it is stuck, but diminishes the effect of the control input.  
 Increasing noise increases completion time. 
 %Large amounts of noise caused failures, defined as trials lasting longer than 1000s.  With $W=50$, XX of 20 trials failed.  With $W=100$, XX of 20 trials failed.
The robots have limited force, so increasing the object weight increases completion time.  
Each shape was designed to have the same mass and area.
 Rectangles and squares tend to get stuck in the 90$^\circ$ workspace corners, and cause longer completion times than circles, triangles, and hexagons.






\begin{figure*}
\centering
\renewcommand{\figwid}{0.5\columnwidth}
\begin{overpic}[width =0.45\columnwidth]{SimVeryNum.pdf}\put(1,55){a)}
\end{overpic}
\begin{overpic}[width =0.45\columnwidth]{SimVeryNoise.pdf}\put(1,55){b)}
\end{overpic}
\begin{overpic}[width =0.45\columnwidth]{SimVeryWeight.pdf}\put(1,55){c)}
\end{overpic}
\begin{overpic}[width =0.45\columnwidth]{SimVeryShape.pdf}\put(1,55){d)}

\end{overpic}
\vspace{-0.5em}
\caption{\label{fig:AutoVeryParam}Parameter sweep for a) number of robots, b) different noise values, c) object weight, and d) object shape.  Each bar is labelled with the number of trials. Completion time is in seconds.
%\vspace{-2em}
}
\end{figure*}






%Algorithm \ref{alg:BlockPushing} is an imperfect solution and has a failure mode if the robot swarm becomes multi-modal with modes separated by an obstacle, as shown in Fig.~\ref{fig:Failure}.  In this case, moving toward a corner will never reduce the variance below $\sigma_{min}^2$.


%  The first challenge is to identify when the distribution has become multi-modal.  Measuring just the mean and variance is insufficient to determine if a distribution is no longer unimodal, but if the swarm is being directed to a corner, and the variance does not reduce below $\sigma_{min}^2$, the swarm has become separated. In this case, we must either manipulate with a partial swarm, or run a gathering algorithm.  For the  {\sffamily S}-shaped workspace in this study, an open-loop input that commands the swarm to move in succession \{{\sc west, north, east, south}\} will move the swarm to the bottom right corner.
%This is not true for all obstacle fields. In a {\sffamily T}-shaped workspace, it is not possible to find an open-loop input that will move the entire swarm to the bottom of the {\sffamily T}.  
 
%  Using only the mean and variance may be overly restrictive.  Many heuristics using high-order moments have been developed to test if a distribution is multimodal~\cite{haldane1951simple}.  Often the sensor data itself, though it may not resolve individual robots, will indicate multi-modality.  For instance CCD images reveal clusters of bacteria, and MRI scans show agglomerations of particles~\cite{stuber2007positive}.  This data can be fitted with $k$-means or expectation maximization algorithms, and manipulation could be performed with the nearest swarm of sufficient size.
  









%%%%%%%%%%%%%%%
%%%%%%%%%%%%%%%%%%%%%%%%%%%%%%%%%%%%%%%%%%%%%%%%%%%%%%%%%%%
\section{Conclusion}\label{sec:conclusion}
%%%%%%%%%%%%%%%%%%%%%%%%%%%%%%%%%%%%%%%%%%%%%%%%%%%%%%%%%%%
 \todo{rewrite}
 
    Micro- and nanorobotics have the potential to revolutionize many applications including targeted material delivery, assembly, and surgery.  The same properties that promise breakthrough solutions---small size and large populations---present unique challenges to generating controlled motion. We want to use large swarms of robots to perform manipulation tasks; unfortunately, human-swarm interaction studies as conducted today are limited in sample size, are difficult to reproduce, and are prone to hardware failures. We present an alternative.

This paper first examines the perils, pitfalls, and possibilities we discovered by launching \href{http://www.swarmcontrol.net}{SwarmControl.net}, an online game where players steer swarms of up to 500 robots to complete manipulation challenges. We record statistics from thousands of players, and use the game to explore aspects of large-population robot control. We present the game framework as a new, open-source tool for large-scale user experiments. One surprising result was that humans completed an object manipulation task \emph{faster} when provided with only the mean and variance of the robot swarm than with full-state feedback. Inspired by human operators, this paper next investigates controllers that use only the mean and variance of a robot swarm. We prove that the mean position is controllable, then provide conditions under which variance is controllable.  We next derive automatic controllers for these and a hybrid, hysteresis-based switching control to regulate the first two moments of the robot distribution.  Finally, we employ these controllers as primitives for an object manipulation task and implement all the automatic controllers on 100 kilobots controlled by the direction of a global light source.
    
    
There are many avenues for future work.  Manipulation by large populations of robots is an immature area and there are many open questions. We invite other collaborators to submit their own experiments to SwarmControl.net.
Topics of interest include control with nonuniform flow such as fluid in an artery, gradient control fields like that of an MRI, competitive playing, multi-modal control, and targeted drug delivery in a vascular network.

%\begin{figure}
%\begin{overpic}[width = 0.48\columnwidth]{Worldbrowsing.pdf}\end{overpic}
%\begin{overpic}[width = 0.48\columnwidth]{USbrowsing.pdf}\end{overpic}
%\vspace{-1em}
%\caption{\label{fig:PlayerLocation}Demographic information on game player's location, provided by Google Analytics. Game players from 84 countries and 49 US states visited our site.
%\vspace{-2em}
%}
%\end{figure}

%\paragraph{Site modifications}
%  The current site is optimized for desktop and laptop users, and we currently do not support mobile users. Our IRB allows us to conduct demographic questionnaires, and we will implement these questionnaires in a future release--currently our only source of demographic data is Google Analytics.
%  
%  We are pursuing partnerships to increase the educational content on our website. Our goal is to highlight a variety of the leading micro- and nano-robotics labs and the challenges they are working on.



%\paragraph{Automatic controllers}
%We have compiled a large body of test results.  Our goal is to design automatic controllers using this data. One avenue is to identify the most proficient players and perform inverse optimal control algorithms to learn the cost functions used by the best players.  
%%  demographics questionaires: 


%We will use the video game described above as part of our outreach efforts.  The nature of ensemble control and the manipulation tasks will make this a puzzle game, where the user will need to determine the correct actions, and groups of actions, to accomplish the task.
%Discovering when a task changes from `fun' to `frustrating' is an active problem in game design.  Game companies (\emph{e.g.} ReignDesign Fig.~\ref{fig:Flockwork}) depend heavily on beta-testing to discover the point at which a task becomes frustrating to a user.  A model of what makes a task frustrating would revolutionize the game industry. Our estimate of this difficulty metric for massive manipulation is shown in Fig.~\ref{fig:GameDifficulty}.  ReignDesign has experience and current contracts designing educational games.  We can incorporate real-world physics to simulate robot control at the micro and the nano scale.
%
%Currently, we can use our control theoretic results to determine what tasks are possible. Unfortunately, this gives us no metric of human difficulty -- which tasks are easy for a human pilot.  What tasks should be off-loaded for computer control?  Fortunately \emph{gamification} provides built-in tools to gather this data in a transparent manner with the user's consent by measuring the time and number of actions required to complete each task, and through \emph{leaderboards} \cite{Zichermann2011,Kapp2012}, which rank users based on the efficiency of their solutions.  Key to the success of this endeavor is an engaging story.  We have the genesis of the a game in \cite{Becker2012l}, with game play based on the desire to steer many robots equipped with suction-cup darts in a surprise attack against an older sister. Our hope is to use current micro- and nanorobotics research to create an engaging story users will delight to immerse themselves in.

%In order to understand the difficulty of the tasks, we can measure the time and number of actions required to complete each task, as in Fig.~\ref{fig:Flockwork}.  This information will be collected to maintain a `top scores' list on the website, and help us answer questions about task difficulty.

%hopefully a discussion about flockworks, a successful multi-agent simulation (but actually a mobile app) here....

%http://gamification.org/wiki/Game_Features/Leaderboards

%   - Chemistry, Micro/nano education, molecular dynamics

%%%%%%%%%%%%%%%
\section*{Acknowledgments}
We thank Haoran Zhao, Jarret Lonsford, An Nguyen, and Lillian Lin for help in making structures for the experiments. 
%Withheld for double-blind review
%%%%%%%%%%%%%%%
%% Use plainnat to work nicely with natbib. 
{\footnotesize
%\bibliographystyle{plainnat}
%\bibliographystyle{SageH}
\bibliographystyle{IEEEtran}
\bibliography{IEEEabrv,ShapingSwarmFrictionSharedInput}
}

% Uncomment to add appendix:
%
\documentclass[conference]{IEEEtran}
\usepackage{times}

% numbers option provides compact numerical references in the text. 
\usepackage[numbers]{natbib}
\usepackage{multicol}
\usepackage[bookmarks=true]{hyperref}

\usepackage{bbm}
\usepackage{calc}
\usepackage{url}
\usepackage{hyperref}
\hypersetup{
  colorlinks =true,
  urlcolor = black,
  linkcolor = black
}
\usepackage{graphicx}
\usepackage[cmex10]{amsmath}
\usepackage{bm}
\usepackage{amssymb}
\usepackage{rotating}


%\usepackage{xfrac}
\usepackage{nicefrac}
\usepackage{cite}
\usepackage[caption=false,font=footnotesize]{subfig}
\usepackage[usenames, dvipsnames]{color}
\usepackage{colortbl}
%\usepackage{caption}

%\usepackage{wrapfig}
\usepackage{overpic}
%\usepackage{subfigure}
%\usepackage{textcomp}
\graphicspath{{./pictures/pdf/},{./pictures/ps/},{./pictures/png/},{./pictures/jpg/}}
\usepackage{breqn} %for breaking equations automatically
\usepackage[ruled]{algorithm}
\usepackage{algpseudocode}
%\usepackage{algorithmic}
\usepackage{multirow}
\usepackage{todonotes}

%\newcommand{\todo}[1]{\vspace{5 mm}\par \noindent \framebox{\begin{minipage}[c]{0.98 \columnwidth} \ttfamily\flushleft \textcolor{red}{#1}\end{minipage}}\vspace{5 mm}\par}
% uncomment this to hide all red todos
%\renewcommand{\todo}{}

%% ABBREVIATIONS
\newcommand{\qstart}{q_{\text{start}}}
\newcommand{\qgoal}{q_{\text{goal}}}
\newcommand{\pstart}{p_{\text{start}}}
\newcommand{\pgoal}{p_{\text{goal}}}
\newcommand{\xstart}{x_{\text{start}}}
\newcommand{\xgoal}{x_{\text{goal}}}
\newcommand{\ystart}{y_{\text{start}}}
\newcommand{\ygoal}{y_{\text{goal}}}
\newcommand{\gammastart}{\gamma_{\text{start}}}
\newcommand{\gammagoal}{\gamma_{\text{goal}}}
\providecommand{\proc}[1]{\textsc{#1}}


\newcommand{\ARLfull}{Aero\-space Ro\-bot\-ics La\-bora\-tory }
\newcommand{\ARL}{\textsc{arl}}
\newcommand{\JPL}{\textsc{jpl}}
\newcommand{\PRM}{\textsc{prm}}

\newcommand{\CM}{\textsc{cm}}
\newcommand{\SVM}{\textsc{svm}}
\newcommand{\NN}{\textsc{nn}}
\newcommand{\prm}{\textsc{prm}}
\newcommand{\lemur}{\textsc{lemur}}
\newcommand{\Lemur}{\textsc{Lemur}}
\newcommand{\LP}{\textsc{lp}} 
\newcommand{\SOCP}{\textsc{socp}}
\newcommand{\SDP}{\textsc{sdp}}
\newcommand{\NP}{\textsc{np}}
\newcommand{\SAT}{\textsc{sat}}
\newcommand{\LMI}{\textsc{lmi}}
\newcommand{\hrp}{\textsc{hrp\nobreakdash-2}}
\newcommand{\DOF}{\textsc{dof}}
\newcommand{\UIUC}{\textsc{uiuc}}
%% MACROS


\providecommand{\abs}[1]{\left\lvert#1\right\rvert}
\providecommand{\norm}[1]{\left\lVert#1\right\rVert}
\providecommand{\normn}[2]{\left\lVert#1\right\rVert_#2}
\providecommand{\dualnorm}[1]{\norm{#1}_\ast}
\providecommand{\dualnormn}[2]{\norm{#1}_{#2\ast}}
\providecommand{\set}[1]{\lbrace\,#1\,\rbrace}
\providecommand{\cset}[2]{\lbrace\,{#1}\nobreak\mid\nobreak{#2}\,\rbrace}
\providecommand{\lscal}{<}
\providecommand{\gscal}{>}
\providecommand{\lvect}{\prec}
\providecommand{\gvect}{\succ}
\providecommand{\leqscal}{\leq}
\providecommand{\geqscal}{\geq}
\providecommand{\leqvect}{\preceq}
\providecommand{\geqvect}{\succeq}
\providecommand{\onevect}{\mathbf{1}}
\providecommand{\zerovect}{\mathbf{0}}
\providecommand{\field}[1]{\mathbb{#1}}
\providecommand{\C}{\field{C}}
\providecommand{\R}{\field{R}}
\newcommand{\Cspace}{\mathcal{Q}}
\newcommand{\Uspace}{\mathcal{U}}
\providecommand{\Fspace}{\Cspace_\text{free}}
\providecommand{\Hcal}{$\mathcal{H}$}
\providecommand{\Vcal}{$\mathcal{V}$}
\DeclareMathOperator{\conv}{conv}
\DeclareMathOperator{\cone}{cone}
\DeclareMathOperator{\homog}{homog}
\DeclareMathOperator{\domain}{dom}
\DeclareMathOperator{\range}{range}
\DeclareMathOperator{\sign}{sgn}
\providecommand{\polar}{\triangle}
\providecommand{\ainner}{\underline{a}}
\providecommand{\aouter}{\overline{a}}
\providecommand{\binner}{\underline{b}}
\providecommand{\bouter}{\overline{b}}
\newcommand{\D}{\nobreakdash-\textsc{d}}
%\newcommand{\Fspace}{\mathcal{F}}
\providecommand{\Fspace}{\Cspace_\text{free}}
\providecommand{\free}{\text{\{}\mathsf{free}\text{\}}}
\providecommand{\iff}{\Leftrightarrow}
\providecommand{\subinner}[1]{#1_{\text{inner}}}
\providecommand{\subouter}[1]{#1_{\text{outer}}}
\providecommand{\Ppoly}{\mathcal{X}}
\providecommand{\Pproj}{\mathcal{Y}}
\providecommand{\Pinner}{\subinner{\Pproj}}
\providecommand{\Pouter}{\subouter{\Pproj}}
\DeclareMathOperator{\argmax}{arg\,max}
\providecommand{\Aineq}{B}
\providecommand{\Aeq}{A}
\providecommand{\bineq}{u}
\providecommand{\beq}{t}
\DeclareMathOperator{\area}{area}
\newcommand{\contact}[1]{\Cspace_{#1}}
\newcommand{\feasible}[1]{\Fspace_{#1}}
\newcommand{\dd}{\; \mathrm{d}}
\newcommand{\figwid}{0.22\columnwidth}
\newcommand{\TRUE}{\textbf{true}}
\newcommand{\FALSE}{\textbf{false}}
\DeclareMathOperator{\atan2}{atan2}


\newtheorem{theorem}{Theorem}
\newtheorem{definition}[theorem]{Definition}
\newtheorem{lemma}[theorem]{Lemma}


\pdfinfo{
   /Author (Shiva Shahrokhi, Arun Mahadev, and Aaron T. Becker)
   /Title  (Supplement toAlgorithms For Shaping a Particle Swarm With a Shared Control Input Using Boundary Interaction)
   /CreationDate (D:20160129120000)
   /Subject (Simple Robots)
   /Keywords (Robots;Uniform Control Inputs)
}

\begin{document}

% paper title
\title{\huge{ \emph{Supplement to} 
Algorithms For Shaping a Particle Swarm\\ With a Shared Control Input Using Boundary Interaction}}

\author{Shiva Shahrokhi, Arun Mahadev, and Aaron T. Becker}


\maketitle

\begin{abstract}
%Also 
Includes algorithms and equations too lengthy for main paper, but potentially useful for the community.
Also links to videos and demonstration code for the algorithms.

Consider a swarm of agents that are controlled by the same global inputs and have no autonomy. This paper presents algorithms for shaping such swarms in 2D.

This model is common for current micro- and nano-robots, whose small size makes it difficult to perform onboard computation or contain a power and propulsion source. For this reason these robots are usually powered and controlled by global inputs, such as a uniform external electric or magnetic field, and every robot receives exactly the same control inputs.
Due to their small size, large numbers of micro-robots are required to deliver sufficient payloads.
 Nevertheless, these applications require precision control of the shape and position of the robot swarm. Precision control requires breaking the symmetry caused by the global input.  

A promising technique uses collisions with boundary walls to shape the swarm, however, the range of configurations created by conforming a swarm to a boundary wall is limited. This paper describes the set of stable configurations of a swarm in two canonical workspaces, a circle and a square. 

To increase the diversity of configurations, we add boundary interaction to our model.  We provide algorithms using friction with walls to place two robots at arbitrary locations in a rectangular workspace.
Next, we extend this algorithm to place $n$ robots at desired locations. We conclude with efficient techniques to control the covariance of a swarm not possible without wall-friction. Simulations and hardware implementations with 100 robots validate these results.

\end{abstract}

\IEEEpeerreviewmaketitle

\section{Introduction}
This supplement gives overviews of the videos and code in 
\S \ref{sec:Videos}, 
provides the algorithm for $y$ position control of two robots in
\S \ref{sec:2robotWallFriction},
and gives gull analytical models for fluid settling in square-shaped tanks in
\S \ref{sec:fluidInPlanarRegion}.


\section{Supplementary Videos}\label{sec:Videos}
Five videos animate the key algorithms in this paper.

\subsection{Robot Swarm in a Circle under Gravity}
The video \emph{Robot Swarm in a Circle under Gravity} shows the stable configuration of a swarm under a constant global input.  Animated plots show mean, variance, covariance, and correlation for a swarm in a circular workspace.
Full resolution video: \url{https://youtu.be/nPFAjVIOxYc}.
An online demonstration and source code of the algorithm are at \citet{Zhao2016mathematicaSquare}.

\subsection{Distribution of Robot Swarm in Square under Gravity }
The video \emph{Distribution of Robot Swarm in Square under Gravity } shows the stable configuration of a swarm under a constant global input.  Animated plots show mean, variance, covariance, and correlation for a swarm in a square workspace.
Full resolution video: \url{https://youtu.be/ZEksDxLpAzg}.
An online demonstration and source code of the algorithm are at \citet{Zhao2016mathematica}.


\subsection{Steering 2 Particles with Shared Controls Using Wall Friction}
Animates Algs. 1, 2, 3 in Mathematica to show how two robots can be arbitrarily positioned in a square workspace. In this video the desired initial and ending positions of the two robots are manipulated, and the path that the robots should follow is drawn. The video ends with an extreme case where the robots must exchange positions. 
Full resolution video: \url{https://youtu.be/5TWlw7vThsM}.
An online demonstration and source code of the algorithm are at \citet{Shahrokhi2015mathematicaParticle}.

\subsection{Arranging a robot swarm with global inputs and wall friction [discrete] }
An implementation of Alg. 4  in {\sc Matlab} that illustrates how the two robots positioning algorithm is extendable to $n$ robots. In this video all  robots gets the same input, but by exploiting wall friction each robot reaches its goal, the formation "UH".
Full resolution video: \url{https://youtu.be/uhpsAyPwKeI}.
Full code is available at \citet{Arun2015}.
Note that this code uses discretized version of Algorithm 3.  The continuous-movement version is illustrated in Fig.\ref{PositionNrobots.pdf}.
\begin{figure}
\begin{center}
	\includegraphics[width=1.0\columnwidth]{PositionNrobots.pdf}
\end{center}
\vspace{-1em}
\caption{\label{fig:construction2d}
Illustration of Alg.\ \ref{alg:PosControlNRobots}, $n$ robot position control  using wall friction.
}
\end{figure}




\subsection{AutomaticCovControl.mp4}
A closed-loop controller that steers a swarm of particles to a desired covariance,  implemented with a box2D simulator. In this video the green ellipse is the desired covariance ellipse, the red ellipse is the current covariance ellipse of the swarm and the red dot is the mean position of the robots. Robots follow the algorithm to achieve the desired values for $\sigma_{goalxy}$, $\sigma_x^2$ and $\sigma_y^2$.

%%%%%%%%%%%%%%%%%%%%%%%%%%%%%
\section{ Algorithm for generating desired $y$ spacing between two robots using wall friction}\label{sec:2robotWallFriction}
\begin{algorithm}
\caption{GenerateDesired$y$-spacing($s_1,s_2,e_1,e_2,L$)}\label{alg:YControl}
\begin{algorithmic}[1]
\Require Knowledge of starting $(s_1,s_2)$ and ending $(e_1,e_2)$ positions of  two robots. 
$(0,0)$ is bottom corner, $s_1$ is rightmost robot, 
 $L$ is length of the walls. Current position of the robots are $(r_1,r_2)$.
\Ensure   $ r_{1x} - r_{2x}  \equiv s_{1x} - s_{2x} $   %$\Delta y(t) \equiv \Delta y(0)$ 
\State $ \Delta s_y  \gets s_{1y} - s_{2y} $
\State $ \Delta e_y \gets e_{1y} - e_{2y} $
\State $ r_1 \gets s_1$, $ r_2 \gets s_2$
\If {$\Delta e_y < 0 $ }
\State $ m \gets ( L-\max( r_{1y},r_{2y}) ,0)   $ \Comment{Move to top wall}
\Else 
\State  $ m \gets ( -\min( r_{1y},r_{2y}),0 )    $ \Comment{Move to bottom wall}
\EndIf
\State $m  \gets  m + (0, -\min( r_{1x},r_{2x} ))$ \Comment{Move to left}
\State $ r_1 \gets r_1+m$, $ r_2 \gets r_2+m$ \Comment{Apply move}
\If {$\Delta e_y - (r_{1y} - r_{2y} ) > 0 $}
\State $ m \gets (\min(|\Delta e_y - \Delta s_y |, L- r_{1y}), 0)$  \Comment{Move top}
\Else
\State $ m \gets (-\min(|\Delta e_y - \Delta s_y |, r_{1y}), 0)$\Comment{Move bottom}
\EndIf 
\State $m  \gets  m + (0, \epsilon)$ \Comment{Move right}
\State $ r_1 \gets r_1+m$, $ r_2 \gets r_2+m$ \Comment{Apply move}
\State $\Delta r_y = r_{1y} - r_{2y}$
\If {$\Delta r_y \equiv \Delta e_y$} 
\State   $ m \gets (e_{1x}-r_{1x}, e_{1y}-r_{1y})$
\State $ r_1 \gets r_1+m$, $ r_2 \gets r_2+m$ \Comment{Apply move}
\State  \Return $(r_1,r_2)$
\Else   
\State \Return GenerateDesired$y$-spacing($r_1,r_2,e_1,e_2,L$)
\EndIf
\end{algorithmic}
\end{algorithm}



%%%%%%%%%%%%%%%%%%%%%%%%%%%
\section{Calculations for modeling swarm as fluid in a simple planar workspace}\label{sec:fluidInPlanarRegion}
Two workspaces are used, a square and a circular workspace.

\subsection{Square Workspace}
This section provides formulas for the mean, variance,  covariance and correlation of a very large swarm of robots as they move inside a square workplace under the influence of gravity pointing in the direction $\beta$. The swarm is large, but the robots are small in comparison, and together cover an area of constant volume $A$. Under a global input such as gravity, they flow like water, moving to a side of the workplace and forming a polygonal shape. The workspace is 

The range of possible angles for the global input angle $\beta $ is [0,2$\pi $). In this range of angles, the swarm assumes eight different polygonal shapes. The shapes alternate between triangles and trapezoids when the area $A$$<$1/2, and alternate between squares with one corner removed and trapezoids when $A$$>$1/2.

Two representative formulas are attached, the outline of the swarm shapes in \eqref{tab:SquareRobotRegions} and $\bar{x}(\beta,A)$ in \eqref{tab:SquareXMean}.




\begin{figure}[h]
\begin{center}
\includegraphics[width=\columnwidth]{SquarePlotPositions.pdf} 
\caption{A swarm in a square workspace under a constant global input assumes either a triangular or a trapezoidal shape if $A<1/2$.  If $A>1/2$ the swarm is either a squares with one corner removed or a trapezoidal  shape.}
\label{fig:friction}
\end{center}
\end{figure} 

\begin{table*}
\begin{align}
\bar{x}(\beta,A) = A\leq \frac{1}{2}: &\begin{cases}
 -\frac{\tan ^2(\beta )}{24 A}-\frac{A}{2}+1 & 0\leq \beta \leq \tan ^{-1}(2 A)\lor 2 \pi -\tan ^{-1}(2 A)<\beta \leq 2 \pi  \\
 1-\frac{1}{3} \sqrt{2} \sqrt{A \tan (\beta )} & \tan ^{-1}(2 A)<\beta \leq \frac{\pi }{2}-\tan ^{-1}(2 A) \\
 \frac{\cot (\beta )}{12 A}+\frac{1}{2} & \frac{\pi }{2}-\tan ^{-1}(2 A)<\beta \leq \tan ^{-1}(2 A)+\frac{\pi }{2} \\
 \frac{1}{3} \sqrt{2} \sqrt{-A \tan (\beta )} & \tan ^{-1}(2 A)+\frac{\pi }{2}<\beta \leq \pi -\tan ^{-1}(2 A) \\
 \frac{\tan ^2(\beta )}{24 A}+\frac{A}{2} & \pi -\tan ^{-1}(2 A)<\beta \leq \tan ^{-1}(2 A)+\pi  \\
 \frac{1}{3} \sqrt{2} \sqrt{A \tan (\beta )} & \tan ^{-1}(2 A)+\pi <\beta \leq \frac{3 \pi }{2}-\tan ^{-1}(2 A) \\
 \frac{1}{2}-\frac{\cot (\beta )}{12 A} & \frac{3 \pi }{2}-\tan ^{-1}(2 A)<\beta \leq \tan ^{-1}(2 A)+\frac{3 \pi }{2} \\
 1-\frac{1}{3} \sqrt{2} \sqrt{-A \tan (\beta )} & \tan ^{-1}(2 A)+\frac{3 \pi }{2}<\beta \leq 2 \pi -\tan ^{-1}(2 A) \\
\end{cases} \nonumber\\
\frac{1}{2}<A<1:&\begin{cases}
 -\frac{\tan ^2(\beta )}{24 A}-\frac{A}{2}+1 & 0\leq \beta \leq \tan ^{-1}\left(\frac{1}{2},1-A\right)\lor 2 \pi -\tan ^{-1}\left(\frac{1}{2},1-A\right)<\beta \leq 2 \pi  \\
 \frac{2 \sqrt{2} \sqrt{(1-A) \tan (\beta )} (A-1)+3}{6 A} & \tan ^{-1}\left(\frac{1}{2},1-A\right)<\beta \leq \frac{\pi }{2}-\tan ^{-1}\left(\frac{1}{2},1-A\right) \\
 \frac{6 A+\cot (\beta )}{12 A} & \frac{\pi }{2}-\tan ^{-1}\left(\frac{1}{2},1-A\right)<\beta \leq \tan ^{-1}\left(\frac{1}{2},1-A\right)+\frac{\pi }{2} \\
 \frac{-2 \sqrt{2} \sqrt{(A-1) \tan (\beta )} (A-1)+6 A-3}{6 A} & \tan ^{-1}\left(\frac{1}{2},1-A\right)+\frac{\pi }{2}<\beta \leq \pi -\tan ^{-1}\left(\frac{1}{2},1-A\right) \\
 \frac{\tan ^2(\beta )}{24 A}+\frac{A}{2} & \pi -\tan ^{-1}\left(\frac{1}{2},1-A\right)<\beta \leq \tan ^{-1}\left(\frac{1}{2},1-A\right)+\pi  \\
 \frac{2 \sqrt{2} \sqrt{(1-A) \tan (\beta )} (1-A)+6 A-3}{6 A} & \tan ^{-1}\left(\frac{1}{2},1-A\right)+\pi <\beta \leq \frac{3 \pi }{2}-\tan ^{-1}\left(\frac{1}{2},1-A\right) \\
 \frac{1}{2}-\frac{\cot (\beta )}{12 A} & \frac{3 \pi }{2}-\tan ^{-1}\left(\frac{1}{2},1-A\right)<\beta \leq \tan ^{-1}\left(\frac{1}{2},1-A\right)+\frac{3 \pi }{2} \\
 \frac{2 \sqrt{2} \sqrt{(A-1) \tan (\beta )} (A-1)+3}{6 A} & \tan ^{-1}\left(\frac{1}{2},1-A\right)+\frac{3 \pi }{2}<\beta \leq 2 \pi -\tan ^{-1}\left(\frac{1}{2},1-A\right) \\
\end{cases}
 \nonumber \\
A=1: &\frac{1}{2}
\end{align}
\protect\caption{$\bar{x}$ in a unit-square workspace}
\label{tab:SquareXMean}
\end{table*}



\begin{table*}
\tiny
\begin{align}
\text{RobotRegion}(\beta,A)= \nonumber 
A\leq \frac{1}{2}:&
\begin{cases}
 \left(
\begin{array}{cc}
 1 & 0 \\
 1 & 1 \\
 -A-\frac{\tan (\beta )}{2}+1 & 1 \\
 -A+\frac{\tan (\beta )}{2}+1 & 0 \\
\end{array}
\right) & 0\leq \beta \leq \tan ^{-1}(2 A)\lor 2 \pi -\tan ^{-1}(2 A)<\beta \leq 2 \pi  \\
 \left(
\begin{array}{cc}
 1 & 1 \\
 1-\sqrt{2} \sqrt{A \tan (\beta )} & 1 \\
 1 & 1-\sqrt{2} \sqrt{A \cot (\beta )} \\
\end{array}
\right) & \tan ^{-1}(2 A)<\beta \leq \frac{\pi }{2}-\tan ^{-1}(2 A) \\
 \left(
\begin{array}{cc}
 1 & 1 \\
 0 & 1 \\
 0 & -A+\frac{\cot (\beta )}{2}+1 \\
 1 & -A-\frac{\cot (\beta )}{2}+1 \\
\end{array}
\right) & \frac{\pi }{2}-\tan ^{-1}(2 A)<\beta \leq \tan ^{-1}(2 A)+\frac{\pi }{2} \\
 \left(
\begin{array}{cc}
 0 & 1 \\
 \sqrt{2} \sqrt{-A \tan (\beta )} & 1 \\
 0 & 1-\sqrt{2} \sqrt{-A \cot (\beta )} \\
\end{array}
\right) & \tan ^{-1}(2 A)+\frac{\pi }{2}<\beta \leq \pi -\tan ^{-1}(2 A) \\
 \left(
\begin{array}{cc}
 0 & 0 \\
 0 & 1 \\
 A-\frac{\tan (\beta )}{2} & 1 \\
 A+\frac{\tan (\beta )}{2} & 0 \\
\end{array}
\right) & \pi -\tan ^{-1}(2 A)<\beta \leq \tan ^{-1}(2 A)+\pi  \\
 \left(
\begin{array}{cc}
 0 & 0 \\
 0 & \sqrt{2} \sqrt{A \cot (\beta )} \\
 \sqrt{2} \sqrt{A \tan (\beta )} & 0 \\
\end{array}
\right) & \tan ^{-1}(2 A)+\pi <\beta \leq \frac{3 \pi }{2}-\tan ^{-1}(2 A) \\
 \left(
\begin{array}{cc}
 0 & 0 \\
 1 & 0 \\
 1 & A-\frac{\cot (\beta )}{2} \\
 0 & A+\frac{\cot (\beta )}{2} \\
\end{array}
\right) & \frac{3 \pi }{2}-\tan ^{-1}(2 A)<\beta \leq \tan ^{-1}(2 A)+\frac{3 \pi }{2} \\
 \left(
\begin{array}{cc}
 1 & 0 \\
 1-\sqrt{2} \sqrt{-A \tan (\beta )} & 0 \\
 1 & \sqrt{2} \sqrt{-A \cot (\beta )} \\
\end{array}
\right) & \tan ^{-1}(2 A)+\frac{3 \pi }{2}<\beta \leq 2 \pi -\tan ^{-1}(2 A) \\
\end{cases}
 %%%%%%%%%%%%%%%%%%%%%%%%%%%%%%%%%%
\nonumber \\
\frac{1}{2}<A<1:&
\begin{cases}
 \left(
\begin{array}{cc}
 1 & 0 \\
 1 & 1 \\
 (1-A)-\frac{\tan (\beta )}{2} & 1 \\
 (1-A)+\frac{\tan (\beta )}{2} & 0 \\
\end{array}
\right) & 0\leq \beta \leq \tan ^{-1}\left(\frac{1}{2},1-A\right)\lor 2 \pi -\tan ^{-1}\left(\frac{1}{2},1-A\right)<\beta \leq 2 \pi  \\
 \left(
\begin{array}{cc}
 1 & 0 \\
 1 & 1 \\
 0 & 1 \\
 0 & \sqrt{2} \sqrt{(1-A) \cot (\beta )} \\
 \sqrt{2} \sqrt{(1-A) \tan (\beta )} & 0 \\
\end{array}
\right) & \tan ^{-1}\left(\frac{1}{2},1-A\right)<\beta \leq \frac{\pi }{2}-\tan ^{-1}\left(\frac{1}{2},1-A\right) \\
 \left(
\begin{array}{cc}
 0 & 1 \\
 1 & 1 \\
 1 & (1-A)-\frac{\cot (\beta )}{2} \\
 0 & (1-A)+\frac{\cot (\beta )}{2} \\
\end{array}
\right) & \frac{\pi }{2}-\tan ^{-1}\left(\frac{1}{2},1-A\right)<\beta \leq \tan ^{-1}\left(\frac{1}{2},1-A\right)+\frac{\pi }{2} \\
 \left(
\begin{array}{cc}
 1 & 1 \\
 0 & 1 \\
 0 & 0 \\
 1-\sqrt{2} \sqrt{-(1-A) \tan (\beta )} & 0 \\
 1 & \sqrt{2} \sqrt{-(1-A) \cot (\beta )} \\
\end{array}
\right) & \tan ^{-1}\left(\frac{1}{2},1-A\right)+\frac{\pi }{2}<\beta \leq \pi -\tan ^{-1}\left(\frac{1}{2},1-A\right) \\
 \left(
\begin{array}{cc}
 0 & 0 \\
 0 & 1 \\
 -(1-A)-\frac{\tan (\beta )}{2}+1 & 1 \\
 -(1-A)+\frac{\tan (\beta )}{2}+1 & 0 \\
\end{array}
\right) & \pi -\tan ^{-1}\left(\frac{1}{2},1-A\right)<\beta \leq \tan ^{-1}\left(\frac{1}{2},1-A\right)+\pi  \\
 \left(
\begin{array}{cc}
 1 & 0 \\
 0 & 0 \\
 0 & 1 \\
 1-\sqrt{2} \sqrt{(1-A) \tan (\beta )} & 1 \\
 1 & 1-\sqrt{2} \sqrt{(1-A) \cot (\beta )} \\
\end{array}
\right) & \tan ^{-1}\left(\frac{1}{2},1-A\right)+\pi <\beta \leq \frac{3 \pi }{2}-\tan ^{-1}\left(\frac{1}{2},1-A\right) \\
 \left(
\begin{array}{cc}
 1 & 0 \\
 0 & 0 \\
 0 & -(1-A)+\frac{\cot (\beta )}{2}+1 \\
 1 & -(1-A)-\frac{\cot (\beta )}{2}+1 \\
\end{array}
\right) & \frac{3 \pi }{2}-\tan ^{-1}\left(\frac{1}{2},1-A\right)<\beta \leq \tan ^{-1}\left(\frac{1}{2},1-A\right)+\frac{3 \pi }{2} \\
 \left(
\begin{array}{cc}
 0 & 0 \\
 1 & 0 \\
 1 & 1 \\
 \sqrt{2} \sqrt{-(1-A) \tan (\beta )} & 1 \\
 0 & 1-\sqrt{2} \sqrt{-(1-A) \cot (\beta )} \\
\end{array}
\right) & \tan ^{-1}\left(\frac{1}{2},1-A\right)+\frac{3 \pi }{2}<\beta \leq 2 \pi -\tan ^{-1}\left(\frac{1}{2},1-A\right) \\
\end{cases},\nonumber\\
%%%%%%%%%%%%%%%%%%%%%%%%%%%%%%%
A=1:&\left(
\begin{array}{cc}
 1 & 0 \\
 0 & 0 \\
 0 & 1 \\
 1 & 1 \\
\end{array}
\right)
\end{align}
\protect\caption{RobotRegions in a unit-square workspace}
\label{tab:SquareRobotRegions}
\end{table*}


\subsection{Circle Workspace}
The area under a chord of a circle is the area of a sector less the area of the triangle originating at the circle center: 
$A=S(sector)-S(triangle)=1/2 LR-1/2 C(1-h)$, thus
\begin{align}
A=(1/2)\left[LR-c(R-h)\right]
\end{align}
where $L$ is arc length, $c$ is chord length, $R$ is radius and $h$ is height. Solving for $L$ and $C$ gives
\begin{align}
L&=2 \cos ^{-1}(1-h)\\
C&=2\sqrt{h(2-h)}
\end{align}
Therefore the area under a chord is
\begin{align}
\cos ^{-1}(1-h)-(1-h) \sqrt{(2-h) h}
\end{align}

For a circular workspace, with $\beta = 0$, the variance of $x$ and $y$ are:
{\tiny
\begin{align}
&\sigma_x^2(h)=\frac{64 (h-2)^3 h^3}{144 \left(\sqrt{-(h-2) h} (h-1)+\arccos(1-h)\right)^2} +\nonumber\\
&\frac{9 \left(\sqrt{-(h-2) h} (h-1)+\arccos(1-h)\right) \left(\sin \left(4 \arcsin(1-h)\right)+4 \arccos(1-h)\right)}{144 \left(\sqrt{-(h-2) h} (h-1)+\arccos(1-h)\right)^2}
\end{align}}

{\tiny
\begin{align}
\sigma_y^2(h)=
\frac{12 \arccos(1-h)-8 \sin \left(2 \arccos(1-h)\right)+\sin \left(4 \arccos(1-h)\right)}{48 \left(\sqrt{-(h-2) h} (h-1)+\arccos(1-h)\right)}
\end{align}}

For $\beta = 0$, $\sigma_{xy}=0$. These values can be rotated to calculate $\sigma_x^2(\beta,h),\sigma_y^2(\beta,h),$ and $\sigma_{xy}(\beta,h)$.

%%%%%%%%%%%%%%%%%%%%%%%%%%%%%%
\section*{Acknowledgments}
This work was supported by the National Science Foundation under Grant No.\ \href{http://nsf.gov/awardsearch/showAward?AWD_ID=1553063}{ [IIS-1553063]}.

%%%%%%%%%%%%%%%
%% Use plainnat to work nicely with natbib. 
\bibliographystyle{plainnat}
\footnotesize
\bibliography{IEEEabrv,ShapingSwarmFrictionSharedInput}
\end{document}





\end{document}