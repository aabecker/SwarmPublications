
\section{Simulation}\label{sec:simulation}

Two simulations were implemented using wall-friction for position control.  The first controls the position of two robots, the second controls the position of $n$ robots.  

Two additional simulations were performed using wall-friction to control variance and covariance.  The first is an open-loop algorithm that demonstrates the effect of varying friction levels.  The second uses a closed-loop controller to achieve desired variance and covariance values.

\subsection{Position Control of Two Robots}

Algorithm \ref{alg:PosControl2Robots}  was implemented in Mathematica using point robots (radius = $0$). 
An online interactive demonstration and source code of the algorithm are available at \citep{Shahrokhi2015mathematicaParticle}.
  Fig.~\ref{fig:shapeControlMathematica1}  shows  an implementation of this algorithm wiht robot initial positions shown by hollow squares and final positions by circles. 
 Dashed lines show the shortest route if robots could be controlled independently, solid lines the generated path.

   %The path given by our algorithm is shown with solid arrows.
%Each row has five snapshots taken every quarter second. 
%For the sake of brevity axis-aligned moves were replaced with oblique moves that combine 
%jjkjk klklk klklklk klklk lklkl lklklklkl  klklklklk lklklklklk lklklkl klklklkl klklklk klklklk klklk klk klkl
% THES IMAGES DON'T DO THIS ANYMORE 
% $\Delta r_x$ is adjusted to $\Delta e_x$ in the second snapshot at $t = 0.25$. 
% The following frames  adjust $\Delta r_y$ to $\Delta e_y$. 
% $\Delta r_y$ is corrected by $t = 0.75$. 
% Finally, the algorithm moves the robots to their corresponding destinations.

\subsection{Position Control of $n$ Robots}
Alg. \ref{alg:PosControlNRobots}  was simulated in {\sc Matlab} using square block robots with unity width. Code is available at \citep{Arun2015}.
Simulation results are shown in Fig.~\ref{fig:4diagramsplots.pdf} for arrangements with an increasing number of robots,  $n$= [8, 46, 130, 390, 862]. 
The distance moved grows quadratically with the number of robots $n$. A best-fit line $210 n^2 + 1200n-10,000$ is overlaid by the data.

In Fig.~\ref{fig:4diagramsplots.pdf}, the amount of clearance is $\epsilon=1$.
Control performance is sensitive to the desired clearance.  As $\epsilon$ increases, the total distance decreases asymptotically, as shown in Fig.~\ref{fig:graphrobo.pdf}, because the robots have more room to maneuver and fewer $\operatorname{DriftMove}$s are required.
%We observed that the value of �?�, greatly affects the total number of moves of commanded moves on the robots. The smaller the value of �?� the greater is the number of moves required to move a said distance. The second graph plot(fig roboplot) is that of �?� vs �total commanded moves� which is the number of moves the global controller commands onto all the robots. Here we take the pattern 
%�ROBO� and calculate the �total commanded moves� for �?�=[ 0.25,0.375,0.5,0.625,0.75,0.875,1]. We can notice an exponential fall in the total command moves. It can be inferred that there is a tradeoff between area of bounded region and the number of moves. For smaller �?�, smaller is the required area to implement algorithm but the longer is the time taken to complete the algorithm. Fig robo
%Unit of �?� is the ratio of minimum clearance value to the body length of a robot.



%The total commanded distance is plotted, which is approximately -0.15v^3 + 385x^2-35000x+300000
%We compare the total distance moved and commanded  with the  \emph{LAP distance}---the shortest distance according to the Linear Assignment Problem using Manhattan distance.   Because all robots are interchangeable, the LAP distance reduces to \[ \text{LAP} =   \sum_{k=1}^n  \left| f_{k,x}-p_{k,x} \right| +  \left|  f_{k,y}-p_{k,y}  \right| . \]


\begin{figure}
\begin{center}
	\includegraphics[width=\columnwidth]{5diagramsplots2.pdf}
\end{center}
\vspace{-1em}
\caption{\label{fig:4diagramsplots.pdf}
The required number of moves under Alg. \ref{alg:PosControlNRobots}  using wall-friction to rearrange $n$ square-shaped 
robots.  %The plot compares  \emph{Total distance}---the sum of the moves made by every robot, with \emph{LAP distance}---the shortest distance according to the Linear Assignment Problem using Manhattan distance.  
See hardware implementation and simulation at \citep{Arun2015}.
}
\end{figure}

\begin{figure}
\begin{center}
	\includegraphics[width=\columnwidth]{graphrobo.pdf}
\end{center}
\vspace{-1em}
\caption{\label{fig:graphrobo.pdf}
Control performance is sensitive to the desired clearance $\epsilon$.  As $\epsilon$ increases, the total distance decreases asymptotically.
}
\end{figure}

\subsection{Efficient Control of Correlation}
%A set of simulations were conducted to demonstrate the importance of boundary friction.  
This section examines maximum correlation values as a function of  $w,L$ using \eqref{eq:correlationFriction}
from Section  \ref{subsec:ClosedLoopCovarianceControl}. 
The maximum correlation using boundary layer friction $\displaystyle  \max_{w,L} \left( \rho(A,h,w,L) \right)$ can be found by gradient descent, as shown in Fig.~\ref{fig:ControlledCorrelation}. 
For swarms with small area, this method enables generating the full range of correlations $\pm 1$, while the stable configuration method from Section \ref{sec:theory} could only generate correlations $\pm 1/2$.  As  the swarm area $A$ increases above $\approx 0.43$, the stable configuration method is more effective.
Larger boundary layers $h$ enable more control of correlation.
The multimedia attachment shows efficient control of correlation with simulations using the 2D physics engine Box2D~\citep{catto2010box2d} and 144 disc-shaped robots with boundary layer model \eqref{eq:boundarylayerflow}.



\begin{figure}
\begin{center}
	\includegraphics[width=\columnwidth]{ControlledCorrelation.pdf}
\end{center}
\vspace{-1em}
\caption{\label{fig:ControlledCorrelation}
Analytical results comparing maximum correlation $\rho_{xy}$ under the boundary layer friction model of \eqref{eq:correlationFriction} with four boundary layer thicknesses $h$ and the stable triangular configuration \eqref{eq:GravityCorrelation}.
}
\end{figure}

