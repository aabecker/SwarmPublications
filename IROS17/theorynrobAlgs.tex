

\section{Position Control of $n$ Robots Using Boundary Interaction}\label{sec:PostionControlnRobots}
%%%%%%%%%%%%%%%%%%%%%%%%%%%%%%%%%%%%%%%%%%%
The ideas from Alg. \ref{alg:optimalAlg}  can be extended to control the position of $n$ particles using wall friction.
The solution is complete, but not optimal, and requires the starting and final configurations of particles to be disjoint.
The solution described here is an iterative procedure with $n$ loops. 
 The $k^{th}$ loop moves the $k^{th}$ robot from a \emph{staging zone} to the desired position in a \emph{build zone}. 
  All robots move according to the global input, but due to wall friction, at the end of the $k^{\text{th}}$ loop, robots 1 through $k$ are in their desired final configuration in the build zone, and robots $k+1$ to $n$ are in the staging zone. 
   See Fig.~\ref{fig:simulationNrobot} for a schematic of the build and staging zones.

\begin{figure}
\begin{center}
	\includegraphics[width=1.0\columnwidth]{PositionNrobots.pdf}
\end{center}
\vspace{-1em}
\caption{\label{fig:simulationNrobot}
Illustration of Alg.\ \ref{alg:PosControlNRobots}, $n$ robot position control  using wall friction.
}
\end{figure}

Assume an open workspace with four axis-aligned walls with infinite friction.
The axis-aligned build zone of dimension $(w_b, h_b)$ containing the final configuration of $n$ robots must be disjoint from the axis-aligned staging zone of dimension $(w_s, h_s)$  containing the starting configuration of $n$ robots.
 Without loss of generality, assume the build zone  is above the staging zone.  Let $d$ be the diameter of the particles.
Furthermore, there must be at least $\epsilon$ space above the build zone, $\epsilon$ below the staging zone, and $\epsilon + d$ to the left of the build and staging zone.  The minimum workspace is then $(\epsilon + d + \max(w_b,w_s), 2\epsilon + h_s,h_b)$.

The $n$ robots position control algorithm relies on a $\text{\sc DriftMove}(\alpha, \beta, \epsilon,\theta)$ control input, described in Alg.~\ref{alg:DriftMove} and shown in Fig.\  \ref{fig:driftmove}.
For $\theta = 0^\circ$, a drift move consists of repeating a triangular movement sequence $\{ (\beta/2,-\epsilon),(\beta/2,\epsilon),(-\alpha,0)\}$. 
 Any particle touching a top wall moves right $\beta$ units, while every particle not touching the top moves right $\beta-\alpha$.

\begin{figure}
\begin{center}
%\includegraphics[width=.47\columnwidth]{driftmove0.pdf}
%	\includegraphics[width=.47\columnwidth]{driftmove1.pdf}
%	\includegraphics[width=.47\columnwidth]{driftmove2.pdf}
	\includegraphics[width=\columnwidth]{driftmovefullres.pdf}
\end{center}
\vspace{-1em}
\caption{\label{fig:driftmove}
A  $\text{\sc DriftMove}(\alpha, \beta, \epsilon,0^\circ)$ repeats a triangular movement sequence $\{ (\beta/2,-\epsilon),(\beta/2,\epsilon),(-\alpha,0)\}$. At the sequence end, robot $A$ has moved $\beta$ units right, and robot $B$ has moved $\beta-\alpha$ units right.}
\vspace{-1em}
\end{figure}

Let $(0,0)$ be the lower left corner of the workspace, $p_k$ the $x,y$ position of the $k^{th}$ robot, and $f_k$ the final $x,y$ position of the $k^{th}$ robot. Label the robots in the staging zone from left-to-right and bottom-to-top, and the $f_k$ configurations top-to-bottom and right-to-left as shown in Fig.~\ref{fig:construction2d}.

\begin{algorithm}
\caption{PositionControl$n$RobotsUsingWallFriction($k$)}\label{alg:PosControlNRobots}
\begin{algorithmic}[1]
\State Move( $-\epsilon, d/2-p_{ky}$) % move  away from right wall and down till robot k touches bottom


\While{ $p_{kx} > d/2$} 
\State $\text{\sc DriftMove}(\epsilon, \min(p_{kx} - d/2,\epsilon), \epsilon,180^\circ)$    %drift move left until kth robot touches left wall
\EndWhile

\State $m \gets \operatorname{ceil}(\frac{f_{ky}-d/2}{\epsilon})$
\State $\beta \gets \frac{f_{ky}-d/2}{m}$
\State $\alpha \gets \beta - \frac{d/2 - p_{ky}-\epsilon}{m}$
\For{ $m$ iterations}
\State $\text{\sc DriftMove}(\alpha, \beta, \epsilon,90^\circ)$    %move kth robot to f_{ky} and leave the rest in position.
\EndFor

\State Move ($d/2+\epsilon-f_{kx}, 0$)  % move the group to the left until k is in the correct relative x position
\State Move ($f_{kx}-d/2, 0$)  

\end{algorithmic}
\end{algorithm}


\begin{algorithm}
\caption{ {\sc DriftMove}($\alpha,\beta,\epsilon,\theta$)
}\label{alg:DriftMove}
particles touching the wall move $\beta$ units, while particles not touching the wall move $\beta-\alpha$ units.
\begin{algorithmic}[1]
\State $R = \begin{bmatrix} \cos(\theta) & -\sin(\theta) \\
 \sin(\theta) & \cos(\theta)  \end{bmatrix}$
\State {\sc Move}$(R \cdot [\beta/2,-\epsilon]^\top)$ 
\State  {\sc Move}$(R \cdot [\beta/2,\epsilon]^\top)$ 
\State  {\sc Move}$(R \cdot [-\alpha,0]^\top)$ 
\end{algorithmic}
\end{algorithm}


Alg. \ref{alg:PosControlNRobots} proceeds as follows:  
First, the robots are moved left away from the right wall, and down so all robots in $k$'s row touch the bottom wall.
Second, a set of $\operatorname{DriftMove}$s are executed that move all robots in $k$'s row left until $k$ touches the left wall, with no net movement of the other robots.
Third, a set of $\operatorname{DriftMove}$s are executed that move only robot $k$ to its target height and return the other robots to their initial heights. 
Fourth, all robots except robot $k$ are pushed left until robot $k$ is in the correct relative $x$ position compared to robots 1 to $k-1$.
Finally, all robots are moved right until robot $k$ is in the desired target position. Running time is $O(n(w+h))$.



The hardware platform depicted in Fig.~\ref{fig:construction2d} is an assembled practical setup that assumes that $\epsilon= 1$ cm. 
The workspace is a $7\times 7$ cm grid space. 
All particles are 3D-printed plastic whose top is a 1cm diameter cylinder with a narrower base that encapsulates a steel bearing ball.
Wall friction is emulated by a toothed wall design to keep particles from moving out of place while implementing the drift move. 
The workspace boundary is mounted on top of a white sheet of cardboard.
Underneath the cardboard, a grid of 3mm diameter magnets glued with 1 cm spacing to a thin board generates the global control input.
 A video attachment  shows the algorithm at work. 
This discretized setup requires several modifications to Alg.~\ref{alg:PosControlNRobots}.
 In this demonstration, all moves are 1 cm in length.
   All drift moves are an counterclockwise \emph{square} move  of size 1 cm$\times$1 cm. 
   Once the $k^{th}$ roller gets to its designated location in each loop, a correction step is implemented. 
   This correction step increases by two the total number of moves required per particle.
   Fig.~\ref{fig:simulationNrobot} shows there are only 6 stages per particle involved in Alg.~\ref{alg:PosControlNRobots}.
The fixed step algorithm requires 8 stages per particle as shown in Fig.~\ref{fig:construction2d}. 

A significant difference between Alg.~\ref{alg:PosControlNRobots} and the fixed move implementation of it is that Alg.~\ref{alg:PosControlNRobots}
enables placing particles at arbitrary, non-overlapping locations, while the fixed move implementation requires goal locations at the center of grid cells. 

\begin{figure}
\begin{center}
	\includegraphics[width=1.0\columnwidth]{multirobotSliderHardware.pdf}
\end{center}
\vspace{-1em}
\caption{\label{fig:construction2d}
Illustration of Alg.\ \ref{alg:PosControlNRobots}, discretized $n$ robot position control  using wall friction.
}
\end{figure}

